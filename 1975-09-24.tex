\documentclass{article}
\usepackage{amsmath}
\usepackage{amssymb}
\usepackage{gensymb}
\begin{document}
\large\noindent\textbf{Meeting September 24th, 1975}

\textbf{1.} V. I. Arnold ``On the theory of envelopes''

\normalsize1\degree.
Let $f:E\to M$ be a smooth mapping.
Diffeomorphisms $r:E\to E$ and $l:M\to M$ are $f-agreeable$, if $fr=lf$.
Additionally, $l$ and $r$ are called $f-raising$ and $f-lowering$
diffeomorphisms, respectively.

2\degree.
Let an $f$ be a Whitney map, mapping the set
$(x_1,\dots,x_n)$ to the set$(\smash{y_1=x_1^{n+1}+x_2x_1^{n-1}+\dots+x_nx_1},
y_2=x_2,\dots,y_n=x_n)$.
The set of critical values of the map $f$ is called the $swallow's$ $tail$.

Theorem 1.
\textit{Let $g$ be a sprout of a smooth function from
$(y_1,\dots,y_n)$ to 0, $g(0)=0$, $\partial g/\partial y_2(0)\neq0$.
Then $g$ converges to the form of $\pm y_2$ $f-$raising (i.e. preserving the
swallow's tail) diffeomorphism.}

Theorem 2.
\textit{Let $\pi$ be a sprout of a smooth function from
$(x_1,\dots,x_n)$ to 0, $\pi(0)=0$, $\partial\pi/\partial x_1(0)\neq0$.
Then $\pi$ converges to the normal form of $x_1+\phi(f(x))$ $f-$lowering
diffeomorphism.}

Let $F(x,x_{n+1})=(f(x),y_{n+1})$, where $y_{n+1}=x_{n+1}$ and let $f$ be a
Whitney map.

Theorem 3.
\textit{Let $\pi$ be a sprout of a smooth function from
$(x_1,\dots,x_{n+1})$ to 0, $\pi(0)=0$,
$\partial\pi/\partial x_{1,n+1}(0)\neq0$.
Then $\pi$ converges to the normal form of $x_1+x_{n+1}$ $F-$lowering
diffeomorphism.}

3\degree.
\textit{A family in} $M$ is called a diagram $B\leftarrow E\to M$, where
$\pi:E\to B$ is a smooth stratification, and $f:E\to M$ is a smooth mapping.
The $equivalence$ of families is a commutative diagram, whose two lines are
families and whose three columns are diffeomorphisms.
The dimension of the base of $B$ is called the $parameter$ $number$ of the
family.

Let $\Omega_n$ be the space of one parameter families of hypersurfaces in
$\mathbb{R}^n$ (dim $B=1$, dim $E=$dim $M=n$, $E$ is compact).

Theorem 4.
\textit{An open set that is compact everywhere in $\Omega_2$ forms families
that, for every point in $E$, are locally equivalent to one of the following:
1) $f_i=x_i$, $\pi=x_1$; 2) $f_1=x_1^2$, $f_{2,3}$, $\pi=x_1+x_3$, or
$x_3+x_1x_2$, or $x_3+x_1x_3+x_1^3\pm x_1x_2^2$, or $x_1\pm x_2^2\pm x_3^2$;
3) $f_1=x_1^3+x_1x_2$, $f_{2,3}=x_{2,3}$; $\pi=x_1+x_3$, or (*)
$x_1\pm x_3^2+\phi(f_1(x),f_2(x))$, or (*) $x_3\pm x_1^2+x_1\phi(f_1(x),f_2(x))$
; 4) $f_1=x_1^4+x_2x_1^2+x_3x_1$, $f_{2,3}=x_{2,3}$; $\pi=x_1+\phi(f(x))$.}
(Results denoted with (*) can probably be improved upon.)

4\degree.
Theorem 6 (Y. A. Brodskij).
\textit{The differential equation $\Phi(x,y,dy/dx)=0$ for $\Phi$ in general
position converges in the common point of the discriminant manifold in normal
form of $(dy/dx)^2=x$ of the local diffeomorphism plane $(x,y)$.}

Quality research of this equation was performed recently by R. Tom, but did not
yield the normal form.

5\degree.
Proof of theorems 1 to 5 are based on the fact that symmetric vector fields on
the plane $\sigma_1=0$ for a free module on the ring of functions from
$(\sigma_2,\dots,\sigma_n)$ to generators grad$(\sigma_i|\sigma_1=0)$.
\end{document}