\documentclass[12pt]{amsart}
\usepackage{geometry}                % See geometry.pdf to learn the layout options. There are lots.
\geometry{letterpaper}                   % ... or a4paper or a5paper or ... 
%\geometry{landscape}                % Activate for for rotated page geometry
%\usepackage[parfill]{parskip}    % Activate to begin paragraphs with an empty line rather than an indent
\usepackage{graphicx}
\usepackage{amssymb}
\usepackage{amsmath}
\usepackage{epstopdf}
\usepackage{gensymb}
\usepackage{color}
\usepackage{textcomp}
\usepackage{tikz}
\usetikzlibrary{arrows}
\DeclareGraphicsRule{.tif}{png}{.png}{`convert #1 `dirname #1`/`basename #1
.tif`.png}
\newcommand{\st}{$^\text{st}$~}
\newcommand{\nd}{$^\text{nd}$~}
\newcommand{\rd}{$^\text{rd}$~}
\renewcommand{\th}{$^\text{th}$~}
\title{On the First Union-wide Mathematical Student Olympiad}
%\date{}                                           % Activate to display a given date or no date
%\author{V.I.~Arnold, ~~~~ A.P.~Shapiro}
\author{V.I. Arnold, A.A. Kirillov, V.M. Tihomirov, M.A. Shubin}
\begin{document}
\maketitle
On the 23\rd and the 24\th of October, 1974, the third and
final round of the Union-wide olympiad ``The Student and techinco-scientific
progress''.
This was the first Union-wide student olympiad; it was on several different
topics, among them, mathematics.

The third round of the subject olympiad in mathematics was held at the
mechanico-mathematical department at MSU.
To hold the competition, a jury was created from the faculty of the department
led by P.S. Alexandrov.
Many professors and graduate students helped with the creation of the problems
and the grading of the solutions.
The final selection of problems was done by th authors of this article.

There were 18 teams at the olympiad; there were two teams from the RSFSR, one
team from each of the other republics, and one team from each of Moscow and
Leningrad.
In total, 89 students participated in the olympiad.
Let us note that all the given problems were solved by the olympiad's
participants.
Let us present a list of the participants who finished in the top ten:
\begin{enumerate}
\item
S.V. Konjagin (Moscow, MSU, mech.-math., 2\nd year),
\item
A.B. Alexandrov (Leningrad, LSU, math.-mech., 4\th year),
\item
V.V. Sologubov (RSFSR-2, MIPT, 2\nd year),
\item
A.N. Drobotenko (Moldavia, Kishinev Polytech. Inst., 3\rd year),
\item
S.E. Belkin (Moscow, MSU, mech.-math., 3\rd year),
\item
E.M. Matveev (Moscow, MSU, mech.-math., 3\rd year),
\item
A.L. Figotin (Ukraine, Kharkiv Inst., mech.-math., 4\th year),
\item
E.V. Callinen (Leningrad, LSU, 4\th year),
\item
N.V. Kartashev (Ukraine, Kiev Inst., mech.-math., 4\th year),
\item
A.M. Steinberg (RSFSR-1, Ural Inst. mech.-math., 4\th year).
\end{enumerate}

The teams stacked up with Moscow and Leningrad sharing first and second place,
with Ukraine coming in third.

Below are presented the problems that made up the third round of the Union-wide
mathematical olympiad and their solutions.
In total there were 11 problems: five on the first day and six on the second.\\
\hspace{0ex}\\

\centerline{\textbf{1\st day}}

\begin{enumerate}
\item
\textit{Prove the the functions $f_1,\dots,f_n$ are linearly independent if and
only if there exists a set of points $t_1,\dots,t_n$, such that
$\det||f_i(t_j)||_{i,j=1}^n\neq0$.}

\textit{Solution.}
If $\det||f_i(t_j)||_{i,j=1}^n\neq0$, then it is clear that $f_i(t)$ are
linearly independent.
Let us prove the converse by induction.
For $n=1$ everything is clear.
Suppose that the statement is true for $n=k$.
Let us prove the statement for $n=k+1$.
Let us find $k$ points $t_1,\dots,t_k$ such that
$\det||f_i(t_j)||_{i,j=1}^k\neq0$.
Suppose that there is no $k+1$\th point with the needed property.
This means that
$$\det
\begin{array}{||ccc||}
f_1(t_1)&\dots&f_{k+1}(t_1)\\
\dotfill&\dotfill&\dotfill\\
f_1(t_k)&\dots&f_{k+1}(t_k)\\
f_1(t)&\dots&f_{k+1}(t)
\end{array}
\equiv0.$$

Expanding the determinant along the last row we get
$$f_1(t)C_1+\dots+f_{k+1}(t)C_{k+1}\equiv0,$$
where
$$C_{k+1}=\det||f_i(t_j)||_{i,j=1}^k\neq0,$$
by inductive hypothesis.
We have arrived at a contradiction with the problem statement.
Thus, the problem statement is true for $n=k+1$.

This problem was fully solved by 25 people.
12 people did not attempt it.
The rest gave incomplete or incorrect solutions.
\item
\textit{The function f(t) is defined over $[0,+\infty)$ and has a continuous
derivative.
\begin{enumerate}
\item
It is known that $f(t)+f'(t)\to A$ as $t\to+\infty$.
Prove that $f(t)\to A$ as $t\to+\infty$.
\item
It is known that $f(t)-f'(t)\to A$ as $t\to+\infty$.
Is it true that $f(t)\to A$ as $t\to+\infty$?
\end{enumerate}}

\textit{Solution.}
\begin{enumerate}
\item
Subtracting a constant, we can simplify the problem to $A=0$.
Suppose that $f(t)+f'(t)=g(t)$.
Then
$$\hspace{10ex}f(t)=Ce^{\text-t}+e^{\text-t}\int\limits_0^te^\tau g(\tau)
\text d\tau=Ce^{\text-t}+e^{\text-t}\left[\int\limits_0^Te^\tau g(\tau)
\text d\tau+\int\limits_0^te^\tau g(\tau)\text d\tau\right].$$

\quad Thus
$$|f(t)|\leq|C|e^{\text-t}+e^{\text-t+T}\max\limits_{t\geq0}|g(t)+
\max\limits_{t\geq T}|g(t)|,$$
from which the problem statement (a) follows.
\item
It is not true that $f(t)\to A$.
Indeed, if $f(t)=e^t$, then $f(t)-f'(t)=0$, but $e^t$ does not approach 0 as
$t\to+\infty$.
\end{enumerate}

This problem was fully solved by 19 people.
45 managed to solve part (b).
11 people did not attempt the problem.
\item
\textit{The function $f(x)$ is defined over $[0,+\infty)$ and has a continuous
derivative, where
$\displaystyle\int\limits_0^{+\infty}|f'(x)|\text{\normalfont d}x<+\infty$.
Prove that the series $\displaystyle\sum\limits_{n=0}^\infty f(n)$ converges if
and only if the integral
$\displaystyle\int\limits_0^\infty f(x)\text{\normalfont d}x$ converges.}

\textit{Solution.}
From the Newton-Liebnitz formula and the theorem about average values follows
that when $0\leq\theta\leq1$
$$\left|\int\limits_n^{n+\theta}f(x)\text dx-\theta f(n)\right|
\leq\theta\int\limits_n^{n+1}|f'(x)|\text dx$$

So when $0<u<v$
$$\hspace{10ex}\left|\int\limits_u^vf(x)\text dx-\sum\limits_{n=[u]+1}^{[v]-1}
f(n)-\theta_1f([u])-\theta_2f([v])\right|\leq\hspace{30ex}$$
$$\hspace{30ex}\leq\int\limits_{[u]}^{[v]+1}|f'(x)|\text dx,\quad0\leq\theta_j
\leq1\quad(j=1,2).$$
The statement we are looking for follows from here with the application of
Cauchy's criterion.

This problem was solved by 14 participants.
31 participants did not even attempt the problem.
\item
\textit{Evaluate $\displaystyle\int\limits_1^{10}x^x\text dx$ with a relative
error of no more than 1\% }(\textit{it is necessary to give the answer, not to
prove the accuracy of the answer; $\ln10\approx2.3026$}).

\textit{Answer.}
$\displaystyle\int\limits_1^{10}x^x\text dx\approx\frac{10^{10}}{1+\ln10}\approx
0.30\cdot10^{10}.$

Reasoning,
$$\int\limits_1^{10}x^x\text dx=\int\limits_1^8x^x\text dx+\int\limits_8^{10}x^x
\text dx=A+B;\quad A<\int\limits_1^88^x\text dx<\frac{8^8}{\ln8}<10^7;$$
$$B=\int\limits_8^{10}x^x\text dx=\int\limits_8^{10}\frac{\text d(e^{x\ln x})}
{1+\ln x}=\frac{x^x}{1+\ln x}\bigg\rvert_8^{10}+\int\limits_8^{10}
\frac{x^x\text dx}{(1+\ln x)^2x}\approx\hspace{10ex}$$
$$\hspace{10ex}\approx\frac{10^10}{1+\ln10}+\frac{x^x}{(1+\ln x)^3x}
\bigg\rvert_8^{10}\approx\frac{10^{10}}{1+\ln10}\approx0.30\cdot10^{10}$$
(the computation ignores values that give an error of under 1\%).

Only two participants solved this problem: A.B. Alexandrov (Leningrad) and P.E.
Normak (Estonia).
50 people did not attempt to solve this problem.
\item
\textit{Let coordinate axes be chosen in three-dimensional space and let $V_1$
be a convex polyhedron, all of whose vertices lie on lattice points.
Let $V_k$ denote the polyhedron, whose set of radial vectors is the result of
multiplying the set of radial vectors of $V_1$ by $k$.
Let $N(V)$ denote the set of lattice points that lie inside of polyhedron $V$ or
on its surface and let $\mu(V)$ denote the volume of polyhedron $V$.
Prove that}
\begin{equation}
N(V_3)-3N(V_2)+3N(V_1)-1=6\mu(V_1).
\end{equation}

The solution will be carried out by inducting over the dimension.
On a line the following formula is true
\begin{equation}
N(V_1)-1=l(V_1),
\end{equation}
where $l(V_1)$ is the length of the line segment $V_1$, whose endpoints lie on
lattice points.
Let us show that on a plane for a convex polygon $V_1$ with vertices on lattice
points, the following formula is relevant
\begin{equation}
N(V_2)-2N(V_1)+1=2S(V_1),
\end{equation}
where $S(V_1)$ is the area of $V_1$.
From equation (2) it follows that (3) is true for degenerate two-dimensional
polygons--line segments.
From here it immediately follows that if polygon $V_1$ were broken diagonally
into two pieces $V_1'$ and $V_1''$, then
$$N(V_2)-2N(V_1)+1=[N(V_2')-2N(V_1')+1]+[N(V_2'')-2N(V_1'')+1].$$
Everything follows to the proof of (3) for triangles.

Let us put the origin $O$ at a vertex of the lattice triangle $V_1$ (Fig.~1) and
let $A$ and $B$ be the triangle's two other vertices.
Let $OA'B'$ be triangle $V_2$.
From Fig.~1 it is clear that the left side of equation (3) is equal to $N(\Pi)$,
the number of lattice points in the shape $\Pi$, represented by the
parallelogram $OACB$ without sides $AC$ and $BC$.
If $S(\Pi)$ is the area of parallelogram $OACB$ (equal to $2S(V_1)$) then to
prove (3) all that's left to check is that $N(\Pi)=S(\Pi)$.
Let us note that translating by $\Pi$ to various lattice points we can
tessellate the entire plane (without intersections).
Additionally a square with a side length of $M$ is intersected by
$\displaystyle\frac{M^2}{S(\Pi)}+O(M)$ $\Pi$-parallelograms.
The number of lattice points in these parallelograms is equal to
$\displaystyle N(\Pi)\frac{M}{S(\Pi)}+O(M)$.
However the number of lattice points in the square of interest is equal to
$M^2+O(M)$.
Thus, $N(\Pi)=S(\Pi)$, which proves equation (3).

We now move to the three-dimensional case.
From (3) it follows that it is sufficient to prove equation (1) for tetrahedra
where the proof is analogous to the two-dimensional case but is based on a more
complicated drawing (Fig.~2).
From the polyhedron $V_3=OA'B'C'$ we throw out three polyhedra of type $V_2$,
located ``in the corners'' $A'$, $B'$, $C'$, we add their pairwise intersections
(of type $V_1$) and we throw away one extra point $P$.
The remainder is a parallelepiped $\Pi$ with a volume of $6\mu(V_1)$ without
part of the surface.
These parallelepipeds can be used to tessellate all space without intersections.
Thus $N(\Pi)=\mu(\Pi)$, which proves the desired statement.

This problem was only solved by E.P. Turkevich (Ukraine).
Several participants were on the right track.
67 people did not attempt it.\\
\end{enumerate}

\centerline{\textbf{2\nd day}}

\begin{enumerate}
\item
\textit{Let $f(x)$ be a continuous function over $[0,1]$ and let $\phi(x)$ be a
continuous periodic function over $(\text-\infty,+\infty)$ with a period of $T$.
Prove that}
\begin{equation}
\lim_{n\to+\infty}\int\limits_0^1f(x)\phi(nx)\text dx=
\frac1T\int\limits_0^T\phi(t)\text dt\cdot\int\limits_0^1f(x)\text dx.
\end{equation}

\textit{Solution.}
Let $\chi_{[\alpha,\beta]}(x)$ be a function equal to 1 when
$x\in[\alpha,\beta]$ and 0 when $x\notin[\alpha,\beta]$.
Let's prove equation (4) for $f(x)=\chi_{[\alpha,\beta]}(x)$, where
$[\alpha,\beta]\subset[0,1]$.
We get
$$\int\limits_0^1\chi_{[\alpha,\beta]}(x)\phi(nx)\text dx=
\int\limits_\alpha^\beta\phi(nx)\text dx=
\frac1n\int\limits_{\alpha n}^{\beta n}\phi(x)\text dx\to
(\beta-\alpha)\frac1T\int\limits_0^T\phi(x)\text dx$$
as $n\to+\infty$.

To prove (4) in the general case we need to uniformly approximate $f(x)$ with
constant piecewise functions.

This problem was solved by 17 people.
25 people did not attempt it.
\item
\textit{What is the largest number of normals can one draw to an ellipse passing
through a point outside of it?
Describe and draw the set of points outside of the ellipse that the maximum
number of normals pass through.}

\textit{Solution.}
Let us write down the equation for an ellipse using polar coordinates
$x=a\cos t$, $y=b\sin t$.
A normal passing through the point $(a\cos t,b\sin t)$ on an ellipse has the
equation $ax\sin t-by\cos t=(a^2-b^2)\sin t\cdot\cos t$.

The envelope of this family of normals is parametrically defined by attaching
the relation acquired by differentiating the equation by $t$ to the equation:
$$ax\cos t+by\sin t=(a^2-b^2)\cos 2t.$$

Excluding the parameter $t$, we arrive at the equation for an astroid
$$(ax)^{2/3}+(by)^{2/3}=(a^2-b^2)^{2/3}.$$

Let us notice that the transition from one number of normals to another can only
happen on this astroid.
From here it is easy to verify that from any point that lies inside the astroid
we can draw 4 normals to the ellipse, from a point outside of it or on the
vertices of the astroid--2 normals, on the astroid (discounting the vertices)--3
normals.

If $a^2>2b^2$, then the astroid goes outside the boundaries of the ellipse
(Fig.~3).
In Fig.~3, the shaded area was required to be displayed by the problem.

This problem was fully solved only by S.E. Belkin (Moscow).
39 students did not attempt the problem.
\item
\textit{Given are two trigonometric polynomials, $p(t)$ and $q(t)$.
Prove that there exists a polynomial $R(x,y)$ such that $R(p(t),q(t))\equiv0$.}

\textit{Solution.}
Let $p(t)$ and $q(t)$ be algebraic polynomials of $e^{it}$ and $e^{\text-it}$
of a degree no larger than $m$.
Let $R(x,y)$ be an algebraic two-variable polynomial of degree $N$.

Then $R(p(t),q(t))$ is an algebraic trigonometric polynomial of $e^{it}$ and
$e^{\text-it}$ of a degree no larger than $mN$.
If we were to take the polynomial $R(x,y)$ with undefined coefficients and set
the coefficients of $e^{ikt}$ in the expression for $R(p(t),q(t))$ to zero, then
we would get $2mN+1$ linear homogeneous equations relative to the
$\displaystyle\frac12(N+1)(N+2)$ coefficients of the polynomial $R$.
It follows that for a big $N$ the number of unknowns will be outgrow the number
of equations, from which follows the existence of the polynomial $R$ that we
were looking for.

This problem was solved by 11 people.
51 people did not attempt to solve it.
\item
\textit{We have a series of spheres whose radii converge to 0, but whose total
volume is infinite.
Prove that it is possible to fit a finite number of these spheres inside of a
cube such that they would fill 99\% of its volume.}

\textit{Solution.}
It is sufficient to prove the following statement: there exists a number
$\alpha$, $0<\alpha<1$, such that for any ``vessel'' $G$ (a region with a
piecewise-smooth boundary) it is possible to place a finite number of spheres
from the given sequence into the vessel such that they will fill a volume no
less than $\alpha\mu(G)$, where $\mu(G)$ is the volume of vessel $G$.
In fact, if this is true, then we can place spheres into the initial cube in
such a way that there will remain an empty space of volume $(1-\alpha)\mu_0$
($\mu_0$ is the volume of the initial cube).
Having examined the remaining unfilled space as a new vessel, let us once again
fill an $\alpha$\th of it and so on.
After $n$ steps, the unfilled volume will be $(1-\alpha)^n\quad\mu_0\to0$ as
$n\to+\infty$.

Next, considering that any vessel can be filled with cubes with an arbitrarily
small error, we see that it is sufficient to learn how to fill an $\alpha$\th of
any cube.
Let us present one possible algorithm for such a filling.
Suppose we want to full a cube with side length 2 with spheres.
Let us take a cube with side length one with its centre in the centre of the
initial cube.
Let us sort the spheres in decreasing order by radius and let their radii be
$r_1\geq r_2\geq\dots$.
Let us choose a number $N$ such that $r_N<1/2$ and start to place the
consecutive spheres $N,N+1,\dots$, positioning their centres inside of the unit
cube.
The process will inevitably terminate on some $M$\th step as the total volume of
all of the spheres is infinite.
Suppose that we managed to fit the spheres of radii $r_N,r_{N+1},\dots,r_{N+M}$.
Let us show that the same centres and doubled radii cover the unit cube in its
entirety\dots
Indeed, if we could find a point in the unit cube that did not get covered by
the spheres with doubled radii then that means that its distance to any sphere
is $\geq r_{N+M}$, which means we could have fit one more sphere with its centre
at that point and with radius $r_{N+M+1}\leq r_{N+M}$.
All constructed spheres obviously lie within the cube with side length 2.
This means that if $V$ is the total volume of the spheres we used then $8V>1$
and so $V>1/8$, which comprises $1/64$ of the volume of the cube with side
length 2.

Three people solved this problem: S.E. Belkin (Moscow), S.V. Konjagin (Moscow),
and V.V. Sologubov (RSFSR-2).
60 people did not attempt the problem.
\item
\textit{A vertex of a curve refers to a point that is an extremum of the
curvature.
Prove that any smooth (infinitely differentiable) closed convex curve contains
at least four vertices.}

\textit{Solution.}
If $(x_1(s),x_2(s))$ is the equation for a curve with its natural parameter ($s$
is the length of the path), then, as we know,
$$\ddot x_1(s)=\text-k(s)\dot x_2(s),\quad\ddot x_2(s)=k(s)\dot x_1(s).$$
From here it follows that for any numbers $a_0$, $a_1$, and $a_2$
\begin{equation}
\int\limits_\Gamma(a_0+a_1x_1+a_2x_2)\frac{\text dk}{\text ds}\text ds=
a_0\int\limits_\Gamma\text dk-a_1\int\limits_\Gamma\ddot x_2\text ds+
a_2\int\limits_\Gamma\ddot x_1\text ds=0.
\end{equation}
The curvature $k(s)$ has a $\Gamma$ maximum and minimum.
If they are both singular vertices then by drawing the straight line
$a_0+a_1x_1+a_2x_2=0$, through them we will arrive at a contradiction with
equation (5), since $(a_0+a_1x_1+a_2x_2)\frac{\text dk}{\text ds}$ does not have
a sign in this case.

This same problem also allows the following simple geometric solution (suggested
by S.M. Gusein-Zade).
Let us inscribe a circle $S$ of maximum radius $r$ into curve $\Gamma$ (Fig.~4).
Let $A$ and $B$ be the points at which the curve $\Gamma$ and $S$ touch and so
that there are no other touching points on the arc $AB$.
It is easy to see that $\cup AB\leq\pi$.
Additionally, the curvature of $\Gamma$ at points $A$ and $B$ is no larger than
$r^{\text-1}$ and strictly less than $r^{\text-1}$ at some points near $A$ and
$B$.
Let us translate the arc $AB$ on curve $S$ perpendicular to the line segment
$AB$ in the direction of its bulge by the maximum possible distance such that it
still intersects curve $\Gamma$ (at point $C$).
In this case $C$ is the touching point of the arc and the curve $\Gamma$ and the
curve $\Gamma$ at point $C$ is no less than $r^{\text-1}$.
From here, the existence of at least two local curvature maxima immediately
follows, and thus, of two minima does as well.

Only one person managed to solve this problem: A.N. Drobotenko (Moldavia).
68 people did not attempt it.
\item
\textit{Does there exist a complex polyhedron with 6 vertices and 15 edges in
four-dimensional space?}

\textit{Solution.}
Such a polyhedron exists.
It is sufficient to take any two triangles in planes $(x,y,0,0)$ and
$(0,0,z,u)$ containing the origin and then their convex envelope.
As an example let us take the points
$$A_1=(1,0,0,0),\quad A_2=(\text-1,1,0,0),\quad A_3=(\text-1,\text-1,0,0),$$
$$A_4=(0,0,1,0),\quad A_5=(0,0,\text-1,1),\quad A_6=(0,0,\text-1,\text-1).$$

For a proof that these are the points we were looking for it is sufficient for
any pair if points $A_i$ and $A_j$ to find a linear function
$\alpha x+\beta y+\gamma z+\delta u$, which has larger values at points $A_i$
and $A_j$ than at any others.
It is obvious that it is sufficient to just examine the pairs$A_i$ and $A_j$
where $(i\leq3,j>3)$.
The needed linear forms are easy to find.
For example, $\text-x+y+2z$ for $A_2$ and $A_4$; $\text-x+y-z+u$ for $A_2$ and
$A_5$; $\text-x+y-z-u$ for $A_2$ and $A_6$.

This problem was only solved by A.B. Alexandrov (Leningrad).
59 people did not attempt it.

\textit{Remark.}
The last problem reflects a rather particular case of an interesting phenomenon:
\textit{in four-dimensional space for any $n$ it is possible to find such a
convex polyhedron with $n$ vertices such that from any vertex it is possible to
``see'' any other vertex, meaning that line segments connecting any two vertices
are edges of this polyhedron.}
As an example of such a polyhedron, we can take, for example, the convex
envelope of any $n$ distinct points that lie on the curve
\begin{equation}
x(t)=(t,t^2,t^3,t^4),\quad\text-\infty<t<+\infty.
\end{equation}

This curve has the following amazing property: if $x_1=x(t_1)$ and $x_2=x(t_2)$
are two of its distinct points then there exists a cross-dimensional hyperplane
that passes through them such that all the other points lie strictly on one side
of it.
In fact, let us pass the hyperplane of type $l(x)=c$ ($l$ is the linear form)
through points $x_1$ and $x_2$, parallel to the tangent vectors to the curve at
those points (it is easy to verify that a vector passing parallel to the chord
$x_1x_2$ and these tangent vectors are linearly independent).
Then $l(x(t))-c$ is a fourth-degree polynomial with two multiple roots, $t_1$
and $t_2$, and therefore $l(x(t))-c$ preserves its sign.
\end{enumerate}
\end{document}
