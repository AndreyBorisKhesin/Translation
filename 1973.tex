\documentclass[12pt]{amsart}
\usepackage{geometry}                % See geometry.pdf to learn the layout options. There are lots.
\geometry{letterpaper}                   % ... or a4paper or a5paper or ... 
%\geometry{landscape}                % Activate for for rotated page geometry
%\usepackage[parfill]{parskip}    % Activate to begin paragraphs with an empty line rather than an indent
\usepackage{graphicx}
\usepackage{amssymb}
\usepackage{amsmath}
\usepackage{epstopdf}
\usepackage{gensymb}
\usepackage{color}
\usepackage{textcomp}
\usepackage{tikz}
\usetikzlibrary{arrows}
\DeclareGraphicsRule{.tif}{png}{.png}{`convert #1 `dirname #1`/`basename #1 .tif`.png}
\title{The Beginning of a New Style of Scientific Literature}
%\date{}                                           % Activate to display a given date or no date
%\author{V.I.~Arnold, ~~~~ A.P.~Shapiro}
\author{Professor V.I.~Arnold\and Academic Y. B. Zeldovich}
\begin{document}
\maketitle
\centerline{Moscow}
\vspace{2em}
\noindent\textbf{V.V. Beletsky.} Essays on the Motion of Celestial Bodies.
M., ``Science'', The main editorial office of physico-mathematical literature,
1972, 360 p.

``Dear Fagotto, can you please start by showing us something simple''--with this
epigraph from M. Bulgakov's ``Master and Margarita'' begins the first essay of
the reviewed book.
The reader is not fooled--the beginning is indeed about several well-known, and
thus simple, classic results.
Having assessed the unusual for a science book writing style from the very first
lines (more precisely, from the brilliant dedication to a private in the
Austro-Hungarian army, and then a warrior in Ferry Bren's international Red
Guard regiment), the reader will slowly start to give due to the unexpected
novelty of the results.
Having turned over the last page, he will surely agree with the reviewers that
the appearance of books such as the one being reviewed and specialists such as
the author of this book is a reflection of the stunning revival that celestial
mechanics is going through under the influence of requests from the theory of
spaceflight.

Here is the repetition of a pattern that we do not encounter in other areas of
knowledge.
It seems that the question of the motion of a body in Earth's gravitational
field or in the Solar system is well-studied.
However, a practical implementation of the process creates the necessity for
more detailed and precise knowledge in the development of optimal courses of
action.
At first glance, the clarity and beauty of simplified theories is lost.
But the next stage new points of view, theories describing the results of more
precise calculations, and new beauty appears.
It is precisely this stage that is reflected in V.V. Beletsky's book.

Like the author fairly remarks, the names of practical specialists, whose
successes in the development of celestial equipment and studies of outer space
stimulated the progress of many precise sciences, did not stay long on magazine
pages; these people, as a rule, do not write books and rarely defend
dissertations.
But without their efforts, the development of mechanics, reflected in many
dissertations, would be unthinkable and many books would not have been written,
among them, ``Essays on the Motion of Celestial Bodies''.
Which is why the all-too familiar words, ``This book is a modest attempt to pay
a tribute of thanks to my teachers, friends, and colleagues'', through the lips
of famous specialist in the mechanics of spaceflight V.V. Beletsky have a
special meaning.

To give a very rough idea of the richness of the contents and the character of
the book, let us mention the names of the essays (and where needed, briefly
explain them) and the epigraphs to several of them.\\

\noindent\textbf{1. On the unperturbed and perturbed motion of a satellite, with
a digression on asymptotic methods of nonlinear mechanics.}\\

\noindent\textbf{2. On the rebirth of an old problem, or what happens if two
masses are placed at a purely imaginary distance from one another.}

\begin{quote}
\dots and the more he looked at the bell-rope, the more he felt that
he had seen something like it, somewhere else, sometime before\dots\\

A.A. Milne, \textit{The World of Pooh}
\end{quote}

\noindent\textbf{3. Yet another reincarnation of an old problem.}

This essay tells of one extreme variant of a problem about the attraction of two
stationary centres (in which one of the centres is shifted to infinity and is
creating a uniform field).\\

\noindent\textbf{4. Motion of the worlds.}

\begin{quote}
\indent The Old Grey donkey, Eeyore, stood by himself in a thistly corner of the
forest, \dots and thought about things.
Sometimes he thought sadly to himself, ``Why?''\ and sometimes he thought,
``Wherefore?''\ and sometimes he thought, ``Inasmuch as which?''\\

A.A. Milne, \textit{The World of Pooh}
\end{quote}

The essay is dedicated to problems on the stability of motion, Laplace's theorem
about the stability of the Solar system, results obtained in this area after the
fundamental work by A.N. Kolmogorov.\\

\noindent\textbf{5. The restricted three-body problem, flight to the Moon, and
galactic evolution.}

\begin{quote}
\indent\textit{Mephisto}: \dots O Moon, for you--my kiss.
(Spreading his dark red cape on the ground, he sits down on it, and flies away
with a whir through the window, which opens widely with a noise).\\

A.V. Lunacharskii, \textit{Faust and the City}
\end{quote}

\noindent\textbf{6. They are waltzing in orbits.}

The essay tells of several notable effects in the rotation and orientation of
satellites\footnote{By the way, in this essay the author suggests making space
probes in the shape of animals that move around on their noses: Rhinogradentia
Otopteryx volitans, discovered by H. Shtumpcke and described in his monograph
``The Build and Life of Rhinogradentia'' (Stuttgart, 1967).
A more accessible statement about the question of rhinogradentia can be found in
the journal ``Science and Life'', 1963, \textnumero~4.
Quoting H. Stumpcke, V.V. Beletsky indicates that ``the existence of literature
about rhinogradentia is not a sufficient condition for the existence of
rhinogradentia'' (p. 171 and 348).}.\\

\noindent\textbf{7. In a spiral to space.}

\begin{quote}
Imagine a blinding, rigid spiral, ascending into the sky!\\

A. Vosnesenskii, \textit{The Triangular Pear}
\end{quote}

This essay is about the acceleration of a satellite under the influence of
gravity, directed along its orbit.\\

\noindent\textbf{8. The full force of the sun blows in the sails.}

This essay presents scientific commentary to Arthur Clarke's tale, ``Sunny
Evening'', describing a race of cosmic sailboats with solar sails.
The solar wind is actually taken into account when calculating orbits of some
satellites.\\

\noindent\textbf{9. The gravity flier.}

\begin{quote}
\indent\dots He swam, even though it was extremely stupid--swimming in space,
where there is nothing to push off of.
He threw his arms forward and moved his legs the way a swimmer would\dots

V. Michaelov, \textit{Amid the Stars}
\end{quote}

In this essay, while discussing the influence of the varying forces of
attraction resulting from the changes in shape of the attracting body, the
author, by the way, writes: ``Principally speaking, a cosmonaut who fell behind
his satellite and lost his personal engine, could catch up to this satellite by
changing his orbit with centre-oriented pulsations.
He should do the breast stroke!
This way he will have a chance of survival.
But do not ask me how long this will take'' (p. 254).

The correspondence attached to this essay between the author and the readers of
the magazine ``Equipment for Youth'', where a paper about the gravity flier was
published, makes a strong impression on the reader.
One of the readers, not understanding the concept of the gravity flier, writes
indignantly: ``The discussion provided in this paper is based on ignorance and
testifies to the malpractice of assigning academic titles\dots
The clever thoughts of professor and doctor of physico-mathematical sciences
came about as a result of his complete misunderstanding of the physical laws of
celestial mechanics, in which he, apparently, ``swims breast stroke'' (p. 254).\\

\noindent\textbf{10. Interplanetary flights: low thrusts for high goals.}

\begin{quote}
We must send the other planets\\
The gospel of the little Earth!

V. Brusoe, \textit{Childrens' Hopes}
\end{quote}

\noindent\textbf{11. Relative motion of orbiting bodies.}\\

\noindent\textbf{12. Cosmic pinwheel.}

\begin{quote}
\indent--Please, señor,--countered Sancho Panza,--those are not so much giants,
as windmills.

Cervantes, \textit{Don Quixote}
\end{quote}

In the last essay the author presents a problem which would have been
unthinkable in old, accessible mechanics.
We are talking about the unusual evolution of rotation and orientation of
``Proton'' type satellites, combined with small aerodynamic forces.\\

For a solid, scientific monograph, the style of V.V. Beletsky's book is unusual
in many ways.
Without exaggeration it can be said that it marks the beginning of a new style
of scientific literature.
The author sincerely and in detail explains the motives of every calculation,
its difficulties, and the psychological side of the study.
The contains no trace of pompous self-importance or of a hurry to provide
results whilst hiding the methods.
The book is decorated and animated by physico-mamthematical sciences' doctor
I.V. Novozhylov's humourous drawings; some of them have been included here.
Not a boring lecture, but a discussion with a brilliant, knowledgeable, and
intelligent persona--that is the general impression from ``Essays''.
Even a person far removed for cosmic problems would happily peruse the book,
skipping, perhaps, the computations.

Now, obviously, it is essential to examine the book's shortcomings.

Being a physicist, one of the reviewers noticed a major flaw in that the book
contains little physics.
Problems are immediately formulated as a system of Newton's equations in three
Cartesian coordinates.
It is likely that for computation on an electronic machine, this is the most
straightforward approach.
But physicists have developed a whole row of concepts such as energy, momentum,
resonance, parametric resonance.
Each concept has its own (specific or approximate) mathematical meaning.
Machine computations automatically conform to the law of conservation of energy
for forces independent of time.
Knowing mathematics, it is possible to avoid using physics in several
situations.
But do we have to do this when we are talking about displaying and explaining
results?!

Another reviewer discovered that in the interesting and captivating problem
``Leonov and the Plug'' (which, by the way, could improve any course about
differential equations), the plug that was thrown to Earth by a cosmonaut falls
behind the satellite, while actually it has to overtake it, as follows from the
law of conservation of momentum (pp. 306--309).
That same reviewer must admit that he was not able to understand the section
``Quantification of Stable Motion''.

These remarks, of course, are not that important, but could be useful for the
making of the second edition of this book, which is absolutely necessary as only
four thousand copies of V.V. Beletsky's ``Essays'' were printed!

The famous French 19$^\text{th}$ century Camille Flammarion published
subsequently ``Astronomy'', ``Astronomy for All'', ``Astronomy for Women''.
Shouldn't V.V. Beletsky and the ``Science'' publishing company follow that same
path when republishing ``Essays''?\\

\textbf{Figure Captions}
\begin{enumerate}
\item
Images by the doctor of physico-mathematical sciences I.V. Novozhilov.
\item
The author of the book uses an integral to get the dynamic trajectories out of
the ``Chest of Oblivion''.
The image contains references to S.Y. Marshak's ``Chest''.
\item
The difficulties of knowledge.
``Reasonable beings placed into existence on a planet in a binary star system
are placed by nature into worse conditions for gaining knowledge than Earth's
mankind''.
\item
The donkey Eeyore is dumbfounded by the ``soul-splitting spectacle'' of the
invariant torus, which frequently appears in the work of one of the reviewers.
\item
The sun fills the sails!
(Spaceflight with the help of a solar sailboat).
\item
A parody of V.V. Beletsky's strange imaginary spaceship--``the gravity flier''.
\end{enumerate}
\end{document}
