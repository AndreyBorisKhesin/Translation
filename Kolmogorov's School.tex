\documentclass{article}
\begin{document}
\title{Kolmogorov's School}
\date{May 27, 1978}
\author{Published in the newspaper "Labour"}
\maketitle
Recently, Andrey Nikolaevich Kolmogorov turned seventy-five.
In the last several years he no longer risks dips in the ice-hole, only allowing himself to bath in snow.
It is doubtful that anyone suspects that the middle-aged, gray-haired man, skiing in nothing but shorts along the forty kilometre trail between the Skalboj and Borej rivers or trekking barefoot to the obscure Black lake through the forest near submoscowian Fryazin in the springtime is one the most important mathematicians of the present day, a Hero of Socialist Labour, and a member of six European and American academies.

Andrey Nikolaevich's road into big mathematics was not easy.
He began making a living himself early on as a train conductor. Kolmogorov was finished school as an external student.
As a student, he continued working.

...Random events are always happening around us.
Movements of molecules and particles weighed in fluid, genetic mutations, failures of components of complicated systems, weather changes and the load of telephone station channels; in forecasting all these processes a large role is played by the theory of probability.

Work done by the academic A. N. Kolmogorov is fundamental to the science of chance.
The application of mathematics to all of these areas allowed for the migration for a purely descriptive approach to well-defined scientific methods.
In a number of cases, mathematical explanations of complicated processes allowed for the establishment of patterns in the many phenomena of the microworld.


\end{document}