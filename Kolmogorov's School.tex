\documentclass{article}
\begin{document}
\title{Kolmogorov's School}
\date{May 27, 1978}
\author{Published in the newspaper "Labour"}
\maketitle
Recently, Andrey Nikolaevich Kolmogorov turned seventy-five.
In the last several years he no longer risks dips in the ice-hole, only allowing himself to bath in snow.
It is doubtful that anyone suspects that the middle-aged, gray-haired man, skiing in nothing but shorts along the forty kilometre trail between the Skalboj and Borej rivers or trekking barefoot to the obscure Black lake through the forest near submoscowian Fryazin in the springtime is one the most important mathematicians of the present day, a Hero of Socialist Labour, and a member of six European and American academies.

Andrey Nikolaevich's road into big mathematics was not easy.
He began making a living himself early on as a train conductor. Kolmogorov was finished school as an external student.
As a student, he continued working.

...Random events are always happening around us.
Movements of molecules and particles weighed in fluid, genetic mutations, failures of components of complicated systems, weather changes and the load of telephone station channels; in forecasting all these processes a large role is played by the theory of probability.

Work done by the academic A. N. Kolmogorov is fundamental to the science of chance.
The application of mathematics to all of these areas allowed for the migration for a purely descriptive approach to well-defined scientific methods.
In a number of cases, mathematical explanations of complicated processes allowed for the establishment of patterns in the many phenomena of the microworld.

Andrey Nikolaevich Kolmogorov was a pioneer and the first explorer in many ares of mathematics.
He is the holder of many notable accomplishments in the theory of probability, the theory of functions, functional analysis, topology, the theory of differential equations, the theory of dynamic systems, the theory of turbulent fluid dynamics etc.
It is difficult to indicate an area of mathematical analysis to which A. N. Kolmogorov did not make a significant contribution or where he did not solve (often two century) old problems.
For the theory of probability and the theory of random processes A. N. Kolmogorov did what Euclid did for geometry: he created a strict axiomatic theory, providing a secure mathematical basis for this area of knowledge.

A. N. Kolmogorov was always working on a wide ring of problems of an applied character.
Among his approximately three hundred published works there is research in mechanics, geology, biology, crystallization of metals etc.

In this century of large scientific teams it seems that the age of large individual advances in science is a thing of the past.
For A. N. Kolmogorov, however, large individual accomplishments are characteristic.
This did not inhibit Andrey Nikolaevich from creating a very large scientific school in Moscow's university, the institution which held an undivided link with all of his activities since 1920.
As a result, many of A. N. Kolmogorov's students created entire scientific fields and created scientific schools, became academics and members of correspondence of the Academy of science.
The number of direct students of Andrey Nikolaevich, having defended their dissertations under his guide, is over sixty, the number of scientific "grandchildren" is in the hundreds, and thousands of scientists experienced the influence of his ideas, lectures, books, and work.

Andrey Nikolaevich put a lot of time and effort into the improvement of mathematical education in the country.
"Truly teaching mathematics well," says A. N. Kolmogorov, "
\end{document}
