\documentclass{article}
\begin{document}
\title{Kolmogorov's School}
\date{May 27, 1978}
\author{Published in the newspaper ``Trud''}
\maketitle
Recently, Andrey Nikolaevich Kolmogorov turned seventy-five.
In the last several years he no longer risks dips in the ice-hole, only allowing himself to bath in snow.
It is doubtful that anyone suspects that the middle-aged, gray-haired man, skiing in nothing but shorts along the forty kilometre trail between the Skalba and Vorya rivers or trekking barefoot to the obscure Black lake through the forest near Fryazin of the Moscow region in the springtime is one the most prominent mathematicians of modern times, a Hero of Socialist Labour, and a member of six European and American academies.

Andrey Nikolaevich's road into big mathematics was not easy.
He began making a living himself early on as a train conductor.
Kolmogorov graduated from school as an external student.
He continued working as an undergraduate student.

...We are constantly surrounded by random events.
Some examples are the motion of molecules and particles weighed in fluid, genetic mutations, failures of components of complicated systems, weather changes and the load of telephone station channels.
Probability theory plays a major role in forecasting all these processes.

The work accomplished by the academician A. N. Kolmogorov is fundamental to the science of chance.
Applications of mathematics to all of these areas allowed the turn from a purely descriptive approach to well-defined scientific methods.
In a number of cases, mathematical explanations of complicated processes allowed for discoveries of patterns in many phenomena of the microworld.

Andrey Nikolaevich Kolmogorov was a pioneer and the first explorer in many areas of mathematics.
He is the author of many outstanding achievements in probability theory, the theory of functions, functional analysis, topology, theory of differential equations, dynamic systems, the turbulence theory for fluids etc.
It is difficult to indicate an area of mathematical analysis to which A. N. Kolmogorov did not make a significant contribution or where he did not solve (often two century) old problems.
For probability theory and the theory of random processes A. N. Kolmogorov accomplished what Euclid did for geometry: he created a rigorous axiomatic theory, providing a robust mathematical foundation for this area of knowledge.

A. N. Kolmogorov was always working on a wide range of applied problems.
Among his approximately three hundred published works one can find studies in mechanics, geology, biology, crystallization of metals etc.

In this century of large scientific teams it seems that the age of large individual advances in science is a thing of the past.
For A. N. Kolmogorov, however, large individual accomplishments are typical.
This did not prevent Andrey Nikolaevich from creating a very large scientific school in Moscow State, the university inseparable from all of his activities since 1920.
Later, many of A. N. Kolmogorov's students themselves created entire scientific fields and new scientific schools, became academicians and corresponding members of the Academy of Science.
Andrey Nikolaevich's students, having defended their dissertations under his supervision, number over sixty, while the number of his scientific ``grandchildren'' is in the hundreds, and thousands of scientists experienced the influence of his ideas, lectures, books, and work.

Andrey Nikolaevich put a lot of time and effort into the improvement of mathematical education in this country.
``Mathematics can be taught truly well,'' says A. N. Kolmogorov, ``only by one who is passionate about it and perceives it as a living, developing science.''
Having possessed the highest degree of this contagious enthusiasm, Andrey Nikolaevich became one of the initiators of creating physical-mathematical boarding schools.
The only kids who can get there are those living in small towns and villages of the country who were winners of mathematical Olympiads.
Many of these ``provincials'' grew up to become big scientists.
For many years A. N. Kolmogorov taught the classes himself to the students there and his pedagogical load can sometimes be as high as twenty-eight hours a week.
Many new programs and textbooks have been created with direct contribution from Andrey Nikolaevich.

...Today, like yesterday and long ago, the academician is full of creative vigour and aspirations.
He does not give himself any concessions.
Scientific work, teaching, all follows the old rhythm.
It seems that Kolmogorov discovered the recipe to eternal youth.

\begin{flushright}
\textbf{V. ARNOLD,\\Professor of the Lomonosov Moscow State University,\\Lenin Prize laureate.}
\end{flushright}
\end{document}
