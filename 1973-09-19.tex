\documentclass{article}
\usepackage{amssymb}
\usepackage{amsmath}
\renewcommand{\d}{\textbf{\$\$\$}}
\begin{document}
\date{}
\author{V. I. Arnold}
\title{Critical Points of Functions and the Classification of Caustics}
\maketitle
\begin{enumerate}
\item
The report was largely dedicated to the overview of the result of papers
[1]$-$[5].
Two theorems that were not included in these papers are described below.
\item
Theorem 1.
(O. V. Ljashko).
\textit{The space of complex Morse functions contains a local homotopic type of
$K(\pi,1)$ space in the neighbourhood of the functions $A_\mu$, $D_\mu$, and
$E_\mu$ (see [2]).
For this $\pi$, the subgroup of finite index $\nu$ is in the Artin braid group
with $\mu$ threads.
Here $\nu=\mu!h^\mu/|W|$, where $h$ is Kokster's value and $|W|$ is the order of
Weil's group.
Thus, $\nu$ has the following values:}

\begin{tabular}[c]{l||c|c|c|c|c}
&&&&&\\Type of feature&$A_\mu$&$D_\mu$&$E_6$&$E_7$&$E_8$\\&&&&&\\\hline&&&&&\\
\multicolumn{1}{c||}{$\nu$}&$(\mu+1)^{\mu-1}$&$2(\mu-1)^\mu$&$2^9\cdot3^4$&
$2\cdot3^12$&$2^2\cdot3^5\cdot5^7$\\&&&&&
\end{tabular}

The space that theorem 1 refers to, $K(\pi,1)$, can be described as follows.
Let $f(x),x\in\mathbb{C}^n$ be a uniformly-weighted polynomial with a critical
point at $x=0$ of type $A_\mu$, $D_\mu$, or $E_\mu$.
Let $F(x,\lambda)=f(x)+\lambda_1e_1+\ldots+\lambda_\mu e_\mu$ be a
$\mu$-parametric versal deformation of the function $f$, where
$\lambda=(\lambda_1,\ldots,\lambda_\mu)\in\mathbb{C}^\mu$ is the parameter and
$e_s(x)$, the terms.

Let $G$ be the set of all values of $\lambda\in\mathbb{C}^\mu$, for which
$F(\cdot,\lambda):\mathbb{C}^n\to\mathbb{C}$ has no multiple critical points and
no mutiple critical values.
$G$ is thus the space $K(\pi,1)$.
\item
Theorem 2.
\textit{Any two-variable holomorphic function with an isolated critical point at
0 with a 3-jet $x^3$ becomes biholomorphic near the point at 0 by coordinate
substitution with one of the normal forms of the following list:}

\begin{tabular}[c]{c}
$J_{k,2}\leftarrow J_{k,3}\leftarrow J_{k,4}\leftarrow\ldots\hspace{19ex}
J_{k+1,2}\leftarrow J_{k+1,3}\leftarrow\ldots$\\
$\downarrow\hspace{45ex}\downarrow\hspace{16ex}$\\
$\ldots\leftarrow J_{k,1}\leftarrow E_{6k}\leftarrow E_{6k+1}\leftarrow E_{6k+2}
\leftarrow J_{k+1,0}\leftarrow J_{k+1,1}\leftarrow E_{6(k+1)}\hspace{11ex}$
\end{tabular}

\textit{Here letters represent the following functions of $x$ and $y$:}
\begin{align*}
E_{6k}&=x^3+y^{3k+1}+a(y)xy^{2k+1},k\geq1;\\
E_{6k+1}&=x^3+xy^{2k+1}+a(y)y^{3k+2},k\geq1;\\
E_{6k+2}&=x^3+y^{3k+2}+a(y)xy^{2k+2},k\geq1;\\
J_{k,l}&=x^3+x^2y^k+a(y)y^{3k+l},k>1,l>0,a_1\neq0;\\
J_{k,0}&=x^3+bx^2y^k+y^{3k}+c(y)xy^{2k+1},k>1,4b^3+27\neq0,
\end{align*}
\textit{where }$a(y)=a_1+a_2y+\ldots+a_{k-1}y^{k-2}$, $c(y)=c_1+c_2y+\ldots+c_{k-2}y^{k-3}$.

The modality of the sprouts of all the indicated functions at 0 is equal to
$k-1$, while the multiple (Milnor's value) is given by $\mu(J_{k,l})=6k+l-2$,
$\mu(E_s)=s$.
\item
I am using the opportuinity to point out several typos in [1]$-$[4].
In [2] on page 13 on lines 24 and 26, and on page 14 on line 27, instead of
$x_1x^4_2$, it should read $ax_1x^4_2$.
In [4] on page 32 on line 5 the bottom should read $-2$ instead of $+2$.
On the same page (page 32 in [4]), diagrams $X_9$ and $J_{10}$ are incorrect.
The correct diagrams are $X_9=T_{2,4,4}$ and $J_{10}=T_{2,3,6}$ and are found on
the next page (page 33 of [4]).

References

[1] V. I. Arnold, Integrals of Rapidly Accelerating Functions and Features of
Projections of Lagrangian Manifolds, Funct. analysis 6:3 (1972), 61$-$62.

[2] V. I. Arnold, Normal Forms of Functions near Degenerate Critical Points,
Weil's groups $A_k$, $D_k$, $E_k$, and Lagrangian Features, Funct. analysis 6:4
(1972), 3$-$25.

[3] V. I. Arnold, Classification of Unimodal Critical Points of Functions,
Funct. analysis 7:3 (1973), 75$-$76.

[4] V. I. Arnold, Notes on Methods of Fixed Phase and Kokster Values, UMN 28:5
(1973), 17$-$44.

[5] V. I. Arnold, Normal Forms of Functions near Degenerate Critical Points,
UMN, 29:2 (1974), 11$-$49.
\end{enumerate}
\end{document}