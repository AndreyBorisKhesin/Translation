\documentclass{article}
\usepackage{amsmath}
\usepackage{amssymb}
\begin{document}
\noindent\textbf{Meeting January 18th, 1977}

\textbf{1.} V. I. Arnold ``On the modern development of I. G. Pertov's work on
the topology of real algebraic varieties''.

The work of I. G. Petrov is fundamental to the study of the topology of real
algebraic manifolds.
In the last several years, there was a significant shift in their study due to
the application of general ideas and methods pertaining to complex analysis and
differential topology.
In particular, I. G. Petrovskii and O. A. Olejnik have established a chain of
evaluations for Eulerian charachterization of real manifolds with an average
Hodge number of its complexification.

Let $A$ be a real, non-trivial hyperplane of even dimension $n$ in the
projective space of $n+1$ dimensions.
V. M. Harmalov generalized the Petrovskii$-$Olejnik inequality to
$|\chi(A)-1|\leq h\smash{^{n/2,n/2}}(CA)-1$, where $\chi$ is the Euler charachteristic
and $h\smash{^{p,q}}$ is the Hodge number (whose dimension is bihomogeneously
comprised of $p$ over $dz$ and $q$ over $d\bar z$ in the group of complex
cohomologies of the classification $CA$ over surface $A$).

On the right side of the Petrovskii$-$Olejnik inequality there was a certain
polynomial of degree $m$ of surface $A$.
Using Hirzebruh's formulas, V. I. Zvonilov verified that this
``Petrovskii$-$Olejnik polynomial'' matches the $h\smash{^{n/2,n/2}}(CA)-1$
term.

The definition of this polynomial, provided by I. G. Petrovskii and O. A.
Olejnik, can be described in the following manner.
Let us examine a cube of side length $m$ in an $(n+1)$-dimensional number space
with its vertices on lattice points.
Let us draw hyperplanes through the vertices of this cube, perpendicular to its
main diagonal.
These hyper planes will split the cube into $n+1$ regions; we will call these
regions \textit{layers}.

The number of lattice points that are strictly inside each of these layers is a
polynomial of $m$.
Let us examine the middle layer (note that there are an odd number of layers,
since $n$ is even).
The number of points strictly inside this layer is the Pertovskii-Olejnik polynomial.

This way, we can use the results of I. G. Petrovskii, O. A. Olejnik, V. M.
Harmalov, and V. I. Zvonilov to show that

\textbf{Theorem. }\textit{The average Hodge number $h\smash{^{n/2,n/2}}$ of a
non-trivial even-dimensional complex projective hyperspace of degree $m$ is
equal to the number of lattice points strictly inside the middle layer in an
$(n+1)$ dimensional cube with a side length of $m+1$.}

This theorem
\end{document}
