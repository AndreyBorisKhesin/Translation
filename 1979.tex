\documentclass[12pt]{amsart}
\usepackage{geometry}                % See geometry.pdf to learn the layout options. There are lots.
\geometry{letterpaper}                   % ... or a4paper or a5paper or ... 
%\geometry{landscape}                % Activate for for rotated page geometry
%\usepackage[parfill]{parskip}    % Activate to begin paragraphs with an empty line rather than an indent
\usepackage{graphicx}
\usepackage{amssymb}
\usepackage{amsmath}
\usepackage{epstopdf}
\usepackage{gensymb}
\usepackage{color}
\usepackage{textcomp}
\usepackage{tikz}
\usetikzlibrary{arrows}
\DeclareGraphicsRule{.tif}{png}{.png}{`convert #1 `dirname #1`/`basename #1 .tif`.png}
\title{Catastrophe Theory}
%\date{}                                           % Activate to display a given date or no date
%\author{V.I.~Arnold, ~~~~ A.P.~Shapiro}
\author{V.I.\ Arnold}
\begin{document}
\maketitle
{\fontsize{9pt}{1em}\selectfont
Vladimir Igorovitch Arnold, a doctor of physical and mathematical sciences, is a
professor at Moscow State University.
He studies the theory of differential equations, classical and celestial
mechanics, singularity theory, real algebraic geometry, and more.
He wrote \textit{Ordinary Differential Equations} (1971), \textit{Mathematical
Methods in Classical Mechanics} (1974), \textit{Additional Chapters to the
Theory of Ordinary Differential Equations} (1978), and co-wrote \textit{Ergodic
Problems in Classical Mechanics} with A.\ Avez (1968).
Arnold is a member of the National Mechanics Committee of the USSR, the vice
president of the Moscow Mathematical Society, an honourary member of the London
Mathematical Society, and a recipient of the Lenin Prize.}\\

Catastrophe theory first appeared in Western print about ten years ago.
Magazines such as ``News Week'' reported the revolution in mathematics as
comparable to the invention of Newton's Calculus.
It was claimed that this new science--catastrophe theory--was much more
useful to humanity than mathematical analysis: while Newton's theories allowed
for the study of smooth, continuous processes, catastrophe theory provided a
universal method of study for all abrupt transitions, gaps, and sudden
noticeable changes.
Hundreds of scientific and near-scientific publications appeared.
They applied catastrophe theory to wildly different objects, for example, the
beating of a heart, geometric and physical optics, embryology, linguistics,
experimental psychology, economics, hydrodynamics, geology, and the theory of
elementary particles.
Among published works about catastrophe theory there are studies about the
effect of alcohol on drivers of various modes of transportation, the sturdiness
of ships, prisoner rebellions, the behaviour of stock market players, modeling
brain activity and mental disorders, and the politics of censorship pertaining
to erotic literature.

In the early 1970's, catastrophe theory quickly became a fashionable, widely
advertised theory.
Its universal pretenses reminded of the pseudo-scientific theories of the
previous century.

Pocket editions of the mathematical papers of the founder of catastrophe theory,
the French mathematician R.\ Thom, were republished and widely circulated.
This had not happened in the mathematical world since the appearance of
cybernetics, which, like catastrophe theory, mostly advertised itself.

After the admirers of catastrophe theory came more sobering, critical papers.
Some of these were printed in publications intended for the layman, bearing
eloquent titles such as ``But he isn't wearing anything at all!''.
Today there are reviews of papers that are critical of catastrophe
theory.\footnote{Couckenheimer, J.\ \textit{The catastrophe controversy}--``The
Mathematical Intelligencer'', 1978, v.\ 1, \textnumero1, p.\ 15-21.}
Today there no unanimity about the value of catastrophe theory.
Below is the personal opinion of the author of the paper.\\

\textbf{\centerline{H.\ Whitney's Singularity Theory}}

In 1955, American mathematician Hassler Whitney published his paper ``Mappings
of the plane into the plane''\footnote{Whitney H.\ Mappings of the plane into the
plane--``Ann.\ Math.'', 1955, v.\ 62, p.\ 374-470.}.
It laid the foundation of a new mathematical theory--singularity theory for
smooth mappings.

To truly understand catastrophe theory, one must first be introduced to elements
of Whitney's singularity theory.
Singularity theory is a grandiose generalization of the study of maxima and
minima of functions.
In Whitney's theory, functions are replaced with mappings, in the simplest case
--mappings of surfaces onto the plane.

The mapping of a surface onto the plane is a matching of every point on the
surface with a point on the plane.
If a point on the surface is defined by coordinates $(x_1,x_2)$ on the surface
and a point on the plane is defined by coordinates $(y_1,y_2)$, the mapping is
defined by a pair of functions:
$$y_1=f_1(x_1,x_2),\quad y_2=f_2(x_1,x_2).$$
Mappings of smooth surfaces onto the plane are all around us.
In fact, most bodies that surround us are bounded by smooth surfaces.
Visible outlines of bodies are just projections of the surfaces surrounding
those bodies onto the eye's retina.
If we were to take a closer look at the bodies that surround us, for example,
the faces of people, we can study the singularities of the visible outlines.

Whitney noticed that in cases of ``general position'', there are only two types
of singularities.
All more complicated singularities are destroyed at the slightest movement of
the bodies or the direction of projection while singularities of these two types
are stable and are preserved despite small deformations in the mapping.

An example of a singularity of the first type (Whitney named it a fold) is the
singularity of a mapping of the projection of a sphere onto the plane at point
on its equator (Fig.\ 1 (Projection of the surface of a sphere onto a horizontal
surface.
A singularity arises on the equator of the sphere, which is called a fold.)).
In local coordinates this singularity is defined by these formulas:
$$y_1=x_1^2,\quad y_2=x_2$$
Unsurprisingly, projecting surfaces of smooth bodies onto an eye's retina
results in such a singularity in the common points.
What is surprising is that other from this singularity, the fold, we regularly
encounter exactly one other singularity, but almost never notice it.

Whitney named this second singularity ``the assembly''.
It results from projecting surfaces shown in Fig.\ 2 (The projection of the
surface $y_1=x_1^3+x_1x_2$ onto the horizontal plane.
There is a singularity called Whitney's assembly at 0.
Three inverses of point $p$, $u$, $v$, and $w$, are shown.) onto the plane.
This surface is defined by the formula $y_1=x_1^3+x_1x_2$ in a space with
coordinates $(x_1,x_2,y_1)$ and is projected onto the horizontal plane
$(y_1,y_2)$.

This way, the mapping is defined in local coordinates by the formulas:
$$y_1=x_1^3+x_1x_2,\quad y_2=x_2.$$

On the horizontal plane of projection forms a half-cubic parabola with a return
point (the peak) in the initial coordinates.
This curve separates the horizontal plane into two distinct parts.
Points in the smaller part each have three inverses (three points in the surface
map to this point).
Points in the larger part each have one.
Upon approaching the curve from the smaller part of the projection, two inverses
merge and disappear (this point contains a singularity, a fold); upon
approaching the peak, all three inverses merge.

Whitney proved that the assembly is stable, so any similar mapping has a
similar singularity in a corresponding similar point (such a singularity, so
that a deformed mapping in corresponding coordinates, near the indicated point
is defined by the same formulas as the original mapping near the original point).
Whitney also proved that any singularity of a smooth mapping of a surface onto
the plane, if disturbed, will promptly ``collapse'' into folds and assemblies.

In this manner, visible contours of smooth bodies in general position contain
return points in places where the projections contain assemblies and do not
contain other singularities: if we look closely, we can find these return points
in the outlines of every face and body.
Let us examine, for example, the surface of a smooth torus (let us say, an
inflated tire).
A torus is normally drawn as depicted in Fig.\ 3 (A standard depiction of a
torus.).
If the torus were transparent, we would see its outline, as depicted in Fig.\ 4
(Singularities of the projection of a torus onto a plane: 4 assembly points are
visible.): the corresponding mapping of a torus onto a plane contains 4 assembly
points.
In this manner, the ends of the lines of the visible outline in Fig.\ 3 are the
return points.
In these points the line of the visible outline contains a half-cubic singularity.

Transparent tori are hard to come by.
Let us examine another transparent body, a bottle.
Fig.\ 5 (Bottleneck.
Two assembly points are visible.) depicts a bottleneck; two assembly points are
visible.
By rocking the bottle, we can convince ourselves that the assemblies are stable.
Thus we obtain convincing experimental confirmation of Whitney's theorem.

After Whitney's pioneering work, singularity theory developed further.
Today, it is one of the central areas of mathematics, where several paths
intersect, linking the most abstract areas of mathematics (differential and
algebraic geometry and topology, group theory, generating mappings,
commutative algebra, complex space theory and so on) with the most applied areas
(theory of motion stability of dynamical systems, theory of bifurcation of
positions of equilibrium, geometric and wave optics and so on).\\

\textbf{\centerline{Applications of Whitney's Theory}}

Since smooth mappings are found ubiquitously, their singularities must be found
everywhere as well.
But since Whitney's theory provides a lot of information about singularities of
mappings in general position, we can try to use this information to study a
larger number of phenomena and processes in all areas of natural history.
This simple idea is the entire essence of catastrophe theory.

In the case where the mapping that is being examined is well-known enough, we
are talking about more or less directly applying the mathematical theory of
singularities to various natural phenomena.
Such an application actually leads to useful results, for example, in elasticity
theory and geometric optics (singularity theory of caustics and wavefronts,
which we will talk more about later).

However, in most works about catastrophe theory, we are usually talking about a
much more debatable situation, when not only is the mapping that is being
studied unknown, its own existence is rather problematic.

Applications of singularity theory in these situations are characteristic of
speculations.
To provide an idea of such mappings, we will recreate the English mathematician
C.\ Zeeman's example of speculative application of Whitney's theory to the study
of activities of a creative character\footnote{Zeeman E.\ C., Catastrophe
theory; a reply to Thom.--In: Dynamical Systems Warwick, 1974, Lecture Notes in
Mathematic 468.\ Berlin--Heidelberg--New York, 1975.\ p.\ 373.}.

We will characterize the creative character (for example, of a scientist) with
three parameters, called ``skill'', ``interest'', and ``achievements''.
Apparently, there must be a correlation between these parameters.
Thus appears a surface in three-dimensional space with coordinates $(S,I,A)$.

Let us project this surface onto the plane $(S,I)$ alone the $A$ axis.
For a surface in general position, the singularities are folds and assemblies
(according to Whitney's theory).
We claim that an assembly, positioned as shown on Fig.\ 6 (A diagram of change of
achievements [$A$] of an artistic character depending on skill [$S$] and
interest [$I$].
With enough interest, increase in skill leads to a significant jump from
``maniac'' to ``genius'', but an increase in interest that is not supported by
an increase in skill leads to the ``genius--maniac'' catastrophe.) satisfactorily
depicts the observed phenomena.

In fact, let us observe how in these suggestions, the achievements of the
scientist will change based on his skill and interest.
If interest is low, then achievements monotonously and fairly slowly rise with
skill.
If interest is sufficiently high, then qualitatively new phenomena occur.
In this case, achievements can rise with a jump as skill increases (such a jump
will happen if, for example, skill and interest change along curve 1 on Fig.\ 6
at point 2).
The area of high achievements, which we finish in in this situation, is labeled
``genii''.

On the other hand, an increase in interest that is not supported with a
corresponding increase in skill, leads to a catastrophe (on curve 3 at point 4,
Fig.\ 6), where achievements fall with a jump and we finish in the area labeled
``maniacs''.
It is instructive that jumps from the state of ``genius'' to the state of
``maniac'' and back happen on different curves.
Thus, at a sufficiently high interest level, a genius and a maniac may have
equal levels of skill and interest, but different achievements (and histories).

The problems with the described model and the set of speculations analogous to
it in catastrophe theory are too obvious to describe in detail.\\

\textbf{\centerline{Catastrophe Machine}}

Unlike the example described above, the application of singularity theory to the
study of the bifurcations of points of balance is perfectly justified.

In many rigid constructions, there are several points of balance in presence of
the same external forces.
Let us examine, for instance, a horizontal ruler whose ends are fixed, that is
weighted down by a weight placed on the midpoint of the ruler.

Along with the point of equilibrium where the ruler is bent downwards by the weight,
another point is possible where is ruler is bent up like a bridge.

As the weight is increased, there comes a point when a ``catastrophe'' or a
``clap'' occurs: the ruler moves from one state to the other.
Singularity theory is applicable to the study of such claps and its predictions
are perfectly supported by experiment.

For a visual illustration of applications of this type, there exist a large
number of inventions.
One of the simplest ones is called Zeeman's catastrophe machine and is depicted
on Fig.\ 7.
(Zeeman's catastrophe machine.
$A$--board; $B$--cardboard disc; $C$--needle that the disc can revolve around;
$D$--needle inserted only into the disc; $E$--needle inserted only into the
board; $F$ and $G$--elastic bands; $H$--pencil; $I$--sheet of paper;
$K$--catastrophe curve.
As pencil $H$ moves, disc $B$ makes jumps when pencil $H$ crosses the
catastrophe curve $K$.)

Everyone can easily make their own catastrophe machine.
For this, one needs a board (on Fig.\ 7--$A$) and, having cut out a cardboard
disc ($B$), insert a needle into the disc's centre ($C$) and into the board so
that the disc can rotate about its axis.
Another needle ($D$) is inserted only into the edge of the disc, and a third
($E$) only into the board.
To complete the assembly of the machine, one only needs two rubber bands ($F$,
$G$), a pencil ($H$), and a sheet of paper ($I$).

After the needle at the edge of the disc has been attached to the motionless
needle and to the pencil with elastic bands, the tip of the pencil is placed at
some point on the sheet of paper, thus stretching the elastic bands.
The disc is affixed in a certain position.
Now, upon moving the tip of the pencil, the disc will rotate.
It turns out, at a certain position of the tip of the pencil, a small change in
said position is capable of producing a ``catastrophe'' i.e.\ the disc will jump
to a new position.
If we marked the position of all of these ``catastrophes'' on the paper, we will
get the ``catastrophe curve'' ($K$).

It turns out that the catastrophe curve itself contains four return points.
Upon crossing the catastrophe curve, a jump may or may not occur, depending on
how the path of the tip of the pencil went around the return points of the
catastrophe curve.

If the reader were to experiment with this machine and tried to find a rule that
determined if there will be a jump, the reader will easily be convinced of the
necessity of a mathematical theory of the phenomenon and could better evaluate
the contribution of the theory to its explanation.

The state of the catastrophe machine is defined by three numbers.
In fact, the position of the tip of the pencil is defined by two coordinates
(called controlling parameters).
The position of the disc is defined by one more number (the angle of the turn),
which is also called the inner parameter of the system.
If we determine all three numbers, we can determine the amount by which the
elastic bands are stretched.
This implies that the potential energy of the entire system can also be
determined.
The disc rotates in such a way as to minimize this energy (at least locally).
For a fixed position of the pencil, the potential energy is a function of the
position of the disc, i.e.\ a function defined over a circle.
This function can have, depending on the values of the defining parameters,
one or several minima (Fig.\ 8a (Dependence of the potential energy of the system
on the controlling and inner parameters.
A jump from one minimum of potential energy to another occurs when one minimum
gets ``eaten'' by an adjacent one.)).
If upon changing the controlling parameters, the minimum changes smoothly, then
a jump does not occur.
A jump occurs at the values of the controlling parameters, for which the local
minimum disappears, merged with the local minimum (Fig.\ 8b).
After the jump, the disc ends up in a position corresponding to the other local
minimum (Fig.\ 8c).

Let us examine the three-dimensional space of the states of this machine.
The states at which the disc is in equilibrium form a smooth surface in this
space.
Let us examine the projection of this surface onto the plane of the controlling
parameters along the axis of the inner parameter (Fig.\ 9 (Diagram of the
three-dimensional space of the sates of the catastrophe machine.
If we were to move from point 1 along path 3, then as we intersect the
catastrophe curve there will be a jump at point 2; if we were move along path 4,
there will be no jump at point 2.)).
This projection contains folds and assemblies.
The projection of the points of the folds is the catastrophe curve.
From Fig.\ 9 we see why changing the controlling parameters across the
catastrophe line sometimes causes, and sometimes does not cause, a jump (this
depends on which sheet of our surface corresponds to the position of the disc).
Using this diagram, we can move from one point on the equilibrium surface to
another without any jumps.

The diagrams of most applications of catastrophe theory are the same as in the
described examples.
We assume that the process in question can be described by several controlling
and inner parameters.
Points of equilibrium if the process create a surface of one or another number
of dimensions in this space.
The projection of the equilibrium surface onto the controlling parameters plane
can have singularities.
We assume that these are singularities in general position.
In this case, singularity theory predicts the geometry of these
``catastrophes'', i.e.\ jumps from one state of equilibrium to another upon a
change in the controlling parameters.

Applications of this type can be more or less justified depending on how
justified the initial packages are.
For example, in the theory of claps of rigid constructions and in the theory of
ship tipping, predictions of the theory are fully confirmed by experiment.
On the other hand, in biology, psychology, and social sciences (let us say, in
applications to the theory of stock buyers' behaviour or the study of nervous
disorders), the initial prepackages, like the conclusions of the theory have a
more heuristic meaning.\\

\textbf{\centerline{Caustics and Wavefronts}}

One of the most important conclusions of the theory is the universality of
several simple images, such as folds, assemblies, and return points, that
should be ubiquitous and that are useful to learn to recognize.
Other than the listed singularities, often we see several more singularities,
that have also received their own names.
Several of these are depicted in Fig.\ 10 (Standard singularities of caustics in
general position in three-dimensional space: swallowtail ($a$), pyramid ($b$),
and wallet ($c$).).

Suppose that a perturbation is propagating in some medium (for example, an
impact wave, light, or an epidemic).

For simplicity, let us start with a flat scenario.
Suppose that in the initial moment the perturbation existed on curve ``$a$''
(Fig.\ 11 (Wavefront, propagating from curve ``$a$''.)) and that its speed of
propagation is equal to 1.
To figure out where the perturbation will be after time $t$, we must place a
segment of length $t$ on every normal to the curve.
The resultant curve is called a wavefront.

Even if the initial wavefront did not contain any singularities, after some time
singularities will start to appear.
For example, upon propagation of a perturbation inside of an ellipse,
singularities shown in Fig.\ 12 (Wavefront propagating inside of an ellipse.)
appear.
These singularities are stable (they are not destroyed by a small change in the
initial front).
With time, given a smooth starting front in general position, only standard
singularities of the same type will form.

With time, only return edges and standard ``swallowtail'' type singularities
appear on a smooth wavefront in general position in three-dimensional space,
depicted in Fig.\ 10a.
But in individual moments, the front can experience other rearrangements, also
of a standard type.
These rearrangements are depicted in Fig.\ 13 (Five standard types of
rearrangements of a wavefront in general position as it moves through
three-dimensional space.).

Right in line with wavefronts, the propagation process is described with the
help of a system of rays.
For example, the spread of perturbations inside of an ellipse can be described
with the help of a system of internal normals to the ellipse (Fig.\ 14 (Internal
normals to an ellipse.)).
This family contains an envelope.
The envelope of a family of rays is called a caustic (i.e.\ ``burning'', since
at these points the light is concentrated).
The caustic is easy to see on the inside surface of a cup lit by sunlight.
The rainbow, which we see in the sky, is similarly described by the caustic of a
system of rays that passed with total internal reflection through a droplet of
water (Fig.\ 15 (Mechanism for the formation of a rainbow.
Rays 1--7 after two refractions and one reflection in a droplet of water form
angles from 0\degree\ to 43\degree\ with respect to the direction of the Sun.
The density of the rays deflected at angles close the maximum possible ones is
higher than that of those deflected at other angles.
The caustic (outside the droplet)--is a line limiting the system of rays; it is
very close to ray 5.)).

The caustic of the elliptical front contains 4 return points.
These singularities are stable: a front close to the ellipse will determine a
caustic with the same singularities.
Caustics in general position in three-dimensional space also only contain
standard singularities.
These singularities are called the swallowtail, the pyramid, and the wallet and
are depicted in Fig.\ 10.

Singularities of the propagating wavefront slide along the caustic.
This phenomenon is easily observable when comparing Fig.\ 12 and Fig.\ 14, as well
as Fig.\ 10 and Fig.\ 13.

Singularities in caustics, fronts, and their rearrangements (Fig.\ 13 and Fig.\
16 (Five standard types of rearrangements of a caustic.
(This diagram is missing in essays on catastrophe theory and is, it seems, being
published for the first time in this paper.))) are well observed in experiments
and today it seems off why this theory did not exist two hundred years ago.
However, the corresponding mathematical apparatus is non-trivial and is tied to
areas of mathematics such as the classification of simple Lie algebras and
Coxter's crystallographic groups; it is also tied (in a fairly mysterious way)
to the classification of regular polyhedra in three-dimensional Euclidian
space.\\

\textbf{\centerline{Catastrophe Theory's Mysticism}}

Applications of singularity theory in natural science does not put a cap on the
directions of catastrophe theory: in line with specific studies such as Zeeman's
work are more philosophic efforts of the mathematician Thom, who first realized
the unifying character of Whitney's work in singularity theory (and preceding
him, work by A.\ Poincaré and A.A.\ Andronov on bifurcation theory) introduced
the term ``catastrophe theory'' and busied himself with its widespread
propaganda.

The characteristic feature of Thom's work in catastrophe theory is his peculiar
style.
Zeeman, a passionate follower of this style, noted that the meaning of Thom's
words only becomes clear when one inserts 99 of his own lines between every pair
of Thom's lines.

To give the reader an opportunity to get an impression of this this style, I
will include here an excerpt from an overview of his perspective of catastrophe
theory, made by Thom in 1974.:

``In the philosophic, metaphysical sense, catastrophe theory cannot bring about
the answers to the big questions worrying mankind.
But in encourages a dialectical, Heraclitian vision of the universe, a vision of
the world as a stage of an endless battle between ``logos'', between archetypes.
Catastrophe theory leads us to a deep polytheistic view: in everything we should
discern the hands of gods.
And here, maybe catastrophe theory will find inevitable limits of its practical
applications.
It will maybe divide the fate of psychoanalysis.
There is no doubt that Freud's main psychological discoveries are true.
And yet knowing these facts has brought little use (when curing mental
disorders).
Just like the hero of the Iliad could not resist the will of a god, say Athena,
we also cannot limit the actions of the archetype without facing him with an
archetype-antagonist, in a battle with an undetermined outcome.
The same reasons that allow us to command our options to act in some situations,
condemn us to impotence in others.
Maybe we will be able to prove the inevitability of some catastrophes, such as
plague or death.
Knowledge will not necessarily be a promise of success or survival; it can also
drive us to determination in our own demise''\footnote{Thom R.\ Catastrophe
theory: Its present state and future perspectives, ibid., p.\ 372.}.

Luckily, singularity theory's excellent results do not depend on catastrophe
theory's mysticism.\\

\noindent\textbf{Recommended Literature}

\noindent Whitney H.\ Mappings of the plane into the plane.--''Ann.\ Math.'',
1955, v.\ 62, p.\ 374--470.

\noindent Poston T., Stewart I.\ Catastrophe Theory and its
Applications.\ Pitman, 1978.\ (Currently being translated into Russian.)

\noindent Thom R.\ Stabilité Structurelle et Morphogénèse.\ N.\ Y., 1972.

\noindent Zeeman E.\ C.\ Catastrophe Theory: Selected Papers
(1972--1977).\ Mass., 1977.

\noindent Brekker T., Lander L.\ Differentiable Germs and Catastrophes.\ M.,
1977.

\noindent Golubitskii M., Giiemin V.\ Stable Mappings and their
singularities.\ M., 1977.
\end{document}
