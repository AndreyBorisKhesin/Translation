\documentclass[12pt]{amsart}
\usepackage{geometry}                % See geometry.pdf to learn the layout options. There are lots.
\geometry{letterpaper}                   % ... or a4paper or a5paper or ... 
%\geometry{landscape}                % Activate for for rotated page geometry
%\usepackage[parfill]{parskip}    % Activate to begin paragraphs with an empty line rather than an indent
\usepackage{graphicx}
\usepackage{amssymb}
\usepackage{amsmath}
\usepackage{epstopdf}
\usepackage{gensymb}
\usepackage{color}
\usepackage{textcomp}
\usepackage{tikz}
\usetikzlibrary{arrows}
\DeclareGraphicsRule{.tif}{png}{.png}{`convert #1 `dirname #1`/`basename #1 .tif`.png}
\title{The loss of stability of auto-oscillations near resonances}
%\date{}                                           % Activate to display a given date or no date
%\author{V.I.~Arnold, ~~~~ A.P.~Shapiro}
\author{V.I.~Arnold}
\begin{document}
\maketitle
\centerline{\textbf{1. Choosing the number of parameters in bifurcation theory}}
\vspace{1em}
The loss of the stability of stationary or auto-oscillating modes is typically
observed as a one-parametric phenomenon.
The results of one-parametric study are well known.
Upon change in parameter (e.g. Reynolds number) the eigenvalues of the
linearized problem, for a certain critical value of the parameter, go to the
minimal axis (or the multipliers of periodic motion--to the unit circle), after
which we can observe either the rapid loss of stability (a transition from the
starting to a distant mode of motion), or a soft one, i.e. appearance of
auto-oscillations for the case of loss of stability of equilibrium and doubling
of the period or the appearance of a two-part quasi-periodic mode for the case
of a loss of periodic motion.
The difference of the motion, appearing upon a gradual loss of stability, from
the initial one is of the order of magnitude as the square root of the
over-criticality (i.e. of the difference of the parameter from the critical
values).

This picture of loss of stability of equilibriums and auto-oscillations was
known from the times of Poincar\'e and was throughly studied by A.A. Andronov and
his school in the 30's.

It turns out that many phenomena that occur from the loss of stability of
auto-oscillations only become clear upon examination of two-parametric and not
one-parametric families of oscillating systems.

The choice of a natural number of parameters in one problem or another is a
problem of the same character as choosing the number of degrees of freedom
when describing a particular physical phenomenon.
Taking additional parameters into account does not lead to significantly
different results if the initial parameters were correctly chosen.

A one-parametric point of view is natural when examining the degeneration of
codimension one, i.e. phenomena, occurring on hypersurfaces (surfaces defined by
one equation) in a functional (more generally speaking, infinite) space of all
the systems of the class in question.
These hypersurfaces can divide the functional space and occur naturally as
boundaries of regions of different behaviours of the system (e.g. boundaries of
the areas on stability etc.).

With a small movement each individual system can be moved off of the degenerate
hypersurface, and in this sense, degenerations do not occur in systems in
general position.

However in the one-parametric family the simplest degenerations occur as a fatal
consequence of the small family.
Indeed, a one-parametric family of systems represents a curve in the functional
space of all systems.
This curve can transversally intersect the hypersurface in functional space
corresponding to the degeneration in question (say, the corresponding loss of
stability).

In this case any nearby curve also intersects our hypersurface (at a point
near the intersection of the initial curve and the hypersurface).
This way, even though it is possible to remove the degeneration for every fixed
value of the parameter, it will still inevitably occur for some nearby value of
the parameter.

While studying one-parametric families, we can neglect the submapping of
codimension two in the functional space of all systems.
Indeed, if the curve is in the functional space representing our one-parametric
family intersects this mapping of codimension two, then an arbitrarily small
movement of this curve can move it off of the mapping.

Mappings of codimension two in functional space correspond to double
degenerations.
The study of such degenerations should be conducted in two-parametric families.
Upon examination of two-parametric families, double degenerations appear as
irremovable in separate points of the plane parameters.
Traces of hypersurfaces corresponding to single degenerations on this plane
divide the regions of the points of double degenerations into sectors.
This division is called a \textit{bifurcation diagram}.
In each sector of the bifurcation diagram, the phase portrait of the system
being examined is of a certain kind.
Upon transitioning from one sector to another, a certain restructuring occurs,
determined by the single degeneration on the boundary of the sector.
In the most special point, the phase portrait degenerates most of all but this
degenerate portrait is actually of least real interest to us: complicated
degenerations should be examined not on their own, but in families with a number
of parameters with which the degenerations become irremovable, and the answer
must be given in the form of totality of bifurcation diagrams (``clock-face'')
and phase portraits, corresponding to sectors, rays, and the centre of the
diagram.

Knowledge of bifurcations in a two-parametric family with a double degeneration
allows us to learn a significant amount about phenomena occurring in
one-parametric families.
Indeed, let us examine a curve in general position on the plane of parameters
in a two-parametric family passing near the special point of double
degeneration.
This line will cross, generally speaking, several sectors of the bifurcation
diagram.
This means that in a one-parametric family corresponding to the given curve,
several subsequent reassemblies will occur, determined by which boundaries of
the sectors the curve intersects.
In this manner, local study of bifurcations in two-parametric families allows us
to obtain information about several global reassemblies in one-parametric
families (for example, information about bifurcations following the first loss
of stability, which is rather difficult to achieve in any other way).\\

\centerline{\textbf{2. Strong and wear resonances}}
\vspace{1em}
In the problem about loss of stability of periodic motion, we can immediately
see one parameter that can be roughly imagined as the module of the
multiplicator\footnote{Multiplicators of periodic motion are eigenvalues of the
linearization of the function of the Poincar\'e series at a point corresponding to
this periodic motion.} (loss of stability occurs upon its transformation to a
unit).

The second parameter in this problem is the argument of the multiplicator:
the multiplicator can exit onto the unit circle at various points along it.
If the argument of the multiplicator at the moment of its exit onto the unit
circle is comparable to $2\pi$, then the spot has a \textit{resonance} upon loss
of stability.
This resonance is a double degeneration and must be studied in a two-parametric
family.
The two corresponding parameters have a simple physical interpretation: one of
them is the decrement and the second is the detuning frequency of the resonance
(the difference between the argument of the multiplicator and its resonance
value).

On the plane of these two parameters appears, as described above, a bifurcation
diagram, whose sectors are responsible for different types of behaviour of the
system.
If in one-parametric family, loss of stability occurs near a resonance then this
one-parametric family is portrayed by a curve on our bifurcation diagram,
passing near the central point of the diagram.
In this case our diagram can be used to read the sequence of reassemblies of
types of motion, occurring near the loss of stability.
By drawing distinct curves on the diagram, we get a finite collection of
sequences of reassemblies, implemented in general one-parametric systems, losing
stability near a resonance.

Calculations show that most of the influence on the picture of loss of stability
is by resonances of degrees $q=1,2,3,4$ (a resonance has degree $q$ if the
multiplicator is equal to $\exp(2\pi ip/q)$.
These resonances are called \textit{strong ILLEGIBLE}, arguments of the
multiplicators of strong resonances ILLEGIBLE $0,\pm90,\pm120,180\degree$.

Resonances of degree $q>4$ are called \textit{weak}.
ILLEGIBLE resonance loss of stability occurs ILLEGIBLE in the non-resonance
case.
From this point of view, ``ILLEGIBLE'' from 360\degree~we should consider only
angles ILLEGIBLE $\pm120,180\degree$, other angles are ``practically not
ILLEGIBLE'' from 360\degree.

The description of bifurcation diagrams and ILLEGIBLE near loss of stability in
the neighbourhood of ILLEGIBLE is provided in later sections.

It turns out that after some reduction (roughly ILLEGIBLE and ``discarding
negligible terms'') ILLEGIBLE to the study of the bifurcation of phase portraits
of vector fields on a plane, that are preserved upon a rotation by $2\pi/q$
and generally depend on two parameters.

The reduced differential equations corresponding to strong resonances have the
form
$$\begin{array}{ll}
\ddot x=\alpha x+\beta+ax^2+bxy,&q=1;\\
\ddot x=\alpha x+\beta y+ax^3+bx^2y,&q=2;\\
\dot z=\epsilon z+Az|z|^2+\overline z^2,&q=3;\\
\dot z=\epsilon z+Az|z|^2+\overline z^3,&q=4.
\end{array}$$
Here $y=\dot x$; $x,y,a,b$ are real and $z,\epsilon,A$ are complex numbers;
the parameters of deformation are $\alpha$, $\beta$, and $\epsilon$; the values
$a,b,\text{Re }A,\text{Im }A$ are non-zero.

A view of bifurcation diagrams and phase portraits is provided in Fig.~1--5.
In the cases of $q=1,2,$ or $3$, the answers depend only on the signs of
$a,b,\text{Re }A,\text{Im }A$.

In the case of q = 4, the plane of the complex variable $A$ is divided into a
series of regions, in each of which the answer (the type of bifurcation diagrams
and reassemblies) is distinct.
The values of parameter $A$ over curves separating these regions do not need to
be examined: they correspond to not a double but a triple degeneration (for the
same reason we do not need to examine the cases
$a=0,b=0,\text{Im }A,\text{Re }A$ where $q\neq4$).

It should be underlined that despite the simplicity of the reduced equations,
their study requires overcoming serious mathematical difficulties (mainly when
studying questions about the number of edge cycles).

The case $q=4$ still has not been fully studied; the pictures of reassemblies
provided below are responsible for just several of the regions on the plane
$A$.\\

\centerline{\textbf{3. Poincar\'e's method of normal forms}}
\vspace{1em}
The foundation of bifurcation theory is Poincar\'e's method of normal forms,
playing a role in the study of non-linear problems that is comparable to the
role Fourier's method (a method of partitioning into natural oscillations) plays
in linear theory.
Even though this method, already developed in Poincar\'e thesis, has been known
for over one hundred years, it only became widely applied to non-linear problems
in the last decade.
Now there are quite a few works where this method is applied to systems with a
finite or with an infinite number of degrees of freedom.
(Unfortunately, in today's time a method for systems of complete spectrum
analogous to Poincar\'e's method of normal forms has not yet been developed.)

The essence of Poincar\'e's method is that instead of searching for solutions of
differential equations, we see diffeomorphisms (replacement variables), that
transforms the equation to possible a simpler form.

Let us examine, for example, a system that has a state of equilibrium near that
state of equilibrium.
Let us choose an origin in this state of equilibrium.
Then the system of equations can be written in the form
\begin{equation}
\dot x=A(x),\quad A(x)=\Lambda x+B(x),
\end{equation}
where $\Lambda$ is the linear operator, $x=(x_1,\dots,x_n)$, and $B$ contains
only terms higher than first degree of $x$.

Let us suppose for simplicity, that all eigenvalues $\lambda_s$ of the $\Lambda$
operator are distinct.
Then we can choose a coordinate system such that the $\Lambda$ operator will be
diagonalized, and in a linear approximation the system falls apart into
one-dimensional ones.

We will seek a replacement of coordinates $x=y+h(y)$, where $h$ means terms
higher than first degree, so that we can try to eliminate terms higher than
first degree on the right side of system (1).
It turns out that in the general (``non-resonance'') case, we can use a
convenient replacement of coordinates to completely eliminate all non-linear
terms in the Taylor series on the right side of the system at zero.

Let us evaluate the right side of (1) in the new system of coordinates.
We get
\begin{align*}
&(E+\partial h/\partial y)\dot y=A(y+h(y)),\\
&\dot y=(E+\partial h/\partial y)^{\text-1}A(y+h(y)).
\end{align*}

And so, $\dot y=\Lambda y+B(y)+C(y)+\dots$, where
$C(y)=\Lambda h(y)-\frac{\partial h}{\partial y}\Lambda y$ (the ellipsis
indicates terms of degree greater than one over $B$ and $h$).

In this manner, \textit{the influence of the deformation of the system of
coordinates converges with the first approximation to the change of the right
part of the equation by the value of the Poisson bracket of the linear part of
the initial field and the field of the deformation}.

Let us examine the terms of the second degree in the Taylor series of the right
side
$$B(x)=B_2(x)+\dots$$
We will seek terms of second degree in the Taylor series in the replacement of
coordinates
$$h(y)=h_2(y)+\dots$$
so that in the new system of coordinates we can eliminate the terms of second
degree of $y$.
We get an equation for $h_2$
$$\Lambda h_2(y)-\frac{\partial h_2}{\partial y}\Lambda y=\text-B_2(y).$$
Let $h_2^{k,m}$ represent (along with $B_2^{k,m}$) the coefficient of the
expansion of the $k^\text{th}$ component of $h$ (along with $B$) where
$y^m=y_1^{m_1}\dots y_n^{m_n}$.
The solution of the equation for $h_2$ is given by
$$h_2^{k,m}=\text-\frac{B_2^{k,m}}{\lambda_k-(m,\lambda)}.$$

The approach of one of the $\lambda_k-(m,\lambda)$ denominators to zero is
called a \textit{resonance}.
The number $|m|=m_1+\dots+m_n$ is called the \textit{degree of the resonance}.

If a system does not have second degree resonances then the terms of second
degree can be eliminated with a convenient replacement of coordinates.

Let us examine the resulting system with the right side of
$\Lambda y+B_3(y)+\dots$ and let us make the replacement $y=z+h_3(z)$.
If there are no third degree resonances we can exclude all third degree terms.

Continuing in this manner, \textit{we can exclude all non-linear terms in the
Taylor series on the right side if the system is devoid of resonances of all
degrees.}
In the new system of coordinates the equation converges to the simplest linear
$\dot w=\Lambda w$.

It should be remarked upon that the converging procedure described above does
not always terminate with a converging series.

In applications of bifurcation theory, a primary role is played by several first
approximations of the method, and the question of convergence does not play a
large part.

If a system has resonances, then several of the non-linear terms are impossible
to kill with replacement of variables, or more specifically, in the presence of
the resonance $\lambda_k=(m,\lambda)$ it is impossible to kill $x^m$ in the
$k^\text{th}$ component of the field.
These terms are called \textit{resonant}.

In this case Poincar\'e's method allows us to eliminate all non-linear terms in
the Taylor series on the right side of the equation, except the resonant ones.

Let us remark that the resonance is one of the types of degenerations, thus, it
must not be present in systems in general position and must be studied not for
an individual system but for a family of systems depending on the parameters.

In a general family of systems depending on the parameters, resonances will be
observed for separate values of the parameter.
In this case, for values of the parameter close to the resonance ones,
Poincar\'e's method allows us to exclude all non-linear terms.

However, the resulting replacements will discontinuously depend on the
parameter, since in the formulas near a resonance value of the parameter there
appears a pole.
Thus while studying the passing through the resonance we should preserve the
resonant terms not only near the resonant value of the parameter (when it is
impossible to eliminate them), but also near nearby non-resonant values of the
parameter.
In this case the replacement of variables can be made smoothly dependent on the
parameter at the moment of passing through the resonance as well.\\

\centerline{\textbf{4. Normal forms near periodic motion}}
\vspace{1em}
Poincar\'e's method, described above for the case of point of equilibrium, can
be reapplied for periodic motion.
There are two functionally equivalent methods to perform this reapplication.

One of these methods consists in examining the function of the Poincar\'e
sequence, i.e. the mapping of the platform, transversally to the cycle in
question, into itself, given by the phase curves of the system in question.
Let us choose a coordinate system on the transversal platform so that the origin
will be in the point of the cycle.

The function of the sequence can be written as
$$x\to\Lambda x+\dots,\quad x=(x_1,\dots,x_n),$$
where $\Lambda$ is the linear operator, and the ellipsis indicates non-linear
terms.
The eigenvalues of the operator $\Lambda$ are called the \textit{multiplicators}
of the periodic motion in question.

With the replacement of coordinates
$$x=y+(\text{non-linear terms})$$
we can eliminate the non-linear terms of the function of the sequence.
The resonances in this case will be the relations $\lambda_k=\lambda^m$, where
$\lambda^m=\lambda_1^{m_1}\dots\lambda_n^{m_n}$.

If there are no resonances then in the Taylor series of the function of the
sequence we can eliminate all non-linear terms.
If there are resonances then we should leave the resonant terms ($y^m$ in the
$k^\text{th}$ component).
When studying families near resonances we should preserve the resonant terms.

The second approach to the problem about periodic motion consists in that phase
curves near the cycle in question are defined with the help of an equation with
periodic coefficients in which the number of phase coordinates if one less than
the initial one, and the role of time is played by the phase of movement along
our cycle.

In this case, Poincar\'e's method results in periodic over time replacements of
variables.
In the non-resonant case these replacements manage to kill all non-linear over
new phase coordinate terms in the expansion of the right side of the Taylor
series over the degrees of the components of the deviation from the cycle in
question (the coefficients of these series are the Fourier series over time).

In the resonant case there are resonant terms.
Suppose that our equation has the form
$$\dot x=\Lambda x+\dots,\quad x=(x_1,\dots,x_n),$$
where the non-linear terms have a period of $2\pi$ over $t$.
Then the condition of resonance has the form $\lambda_k=\lambda^m+im_0$,
the corresponding resonant term contains $x^me^{im_0t}$ in its $k^\text{th}$
component.

Upon studying families we should preserve the summand in normal form not only at
the moment of resonance, but also near it.\\

\centerline{\textbf{5. Preliminary analysis of the problem of loss of stability
in periodic motion}}
\vspace{1em}
Let us apply the above to the problem of loss of stability.
Let us examine the functional space of systems with stable periodic movement.
The boundary of the region of the system with stable periodic movement
represents the hypersurface with three components, each of which has codimension
1 in functional space.

In fact, on the boundary lie systems that have a multiplicator exactly equal to 1.
But the characteristic equation is real, so exactly one multiplicator can be on
the unit circle if it is equal to 1 or -1.
If the multiplicator is not 1 or -1, then there must be at least two complex
conjugate multiplicators on the circle.

It is easy to figure out that in each of the three cases (the multiplicator is
1, -1, or a complex pair) there is a corresponding hypersurface in functional
space, i.e. it is given by one condition on the system.
For systems on the boundary of the region of stability, all other multiplicators
lie strictly inside the unit circle.
In relation with section 1, each of the three cases can be encountered in an
irremovable case in one-parametric families in general position, and we must
study the bifurcations of the parameter passing through the critical value.

In the case where one of the multiplicators is equal to 1, the corresponding
bifurcation is a merger of a stable cycle with an unstable one, after the
critical value both cycles disappear; the loss of the stability meanwhile, is
rough.

When one of the multiplicators is equal to -1, the cycle losing stability does
not disappear.
At a critical value of the parameter, the cycle losing stability is approached
by the passing around it double stable cycle.
This double cycle either is born when passing through the critical value (in
this case the loss of stability is smooth) or dies (in this case the radius of
attraction of the stable cycle becomes small even before the rough loss of
stability).

All of these results easily follow from study from the presented above
Poincar\'e's theorem.
In fact, the function of the sequence with preservation of first resonant terms
has the form
\begin{align*}
&x\to(1+\epsilon)x+Cx^2+\dots\quad(\lambda=+1),\\
&x\to(\text-1+\epsilon)x+Cx^3\quad(\lambda=\text-1).
\end{align*}
Here $x$ represents the coordinate in the eigendirections, corresponding to the
eigenvalue $\lambda$ with a modulus equal to 1.
We represented with $\epsilon$ the parameter of deformation; the coefficient
$C$, generally speaking, depends on $\epsilon$.
The sign of $C(0)$ regulates the birth and death of cycles.
In the directions of the remaining eigenvalues there is a compression so that
they will not influence the bifurcation (see Shoshitajshvili's general theory
[1]).

Let us now address the third case of loss of stability, when a pair of
multiplicators exit onto the unit circle.

The corresponding hypersurface in functional space intersects the previous
hypersurfaces (corresponding to the cases $\lambda=1$ and $\lambda=\text-1$)
over surfaces with codimension two.
The common points on these surfaces correspond to the systems with a double
eigenvalue $\lambda=1$ ($\lambda=\text-1$) and with a Jordanian grid of order 2.
These two cases are realized as irremovable with the availability of two
parameters and must be examined separately.

For the remaining cases of an exit of a pair of eigenvalues of the linearization
of the function of the sequence on the unit circle, let us apply Poincar\'e's
method.
Let us assume for simplicity that there are only two eigenvalues, i.e. that the
initial phase space, where lies the cycle in question, is three-dimensional.

The conclusions, to which the calculations lead to using this method consist of
the following.
\begin{enumerate}
\item The function of the sequence is reduced to normal form which is a
conversion in unit time in the phase flow of the vector field on the surface (at
least on the level of the Taylor expansion of finite degree).
\item The resulting vector field on the plane is invariant relative to the group
of all rotations of the plane in the non-resonant case and relative to the
finite group of rotation by angles that are multiples of $2\pi/q$ in the case of
a resonance of order $q$.
\item The first two conclusions remain relevant in the case when the system is
dependent on a parameter.
A replacement reducing a finite segment of the Taylor series to normal form
smoothly depends on the parameter with and without resonances, if we preserve
the resonant terms with an exact resonance or with near values of the parameter.
Then the symmetry if the normal form relative to rotation by angles that are
multiples of $2\pi/q$, has a place not just with an exact resonance but also
with near values of the parameter.
\item The equation in normal form near a resonance of order $q>2$ has the form
$$\dot z=\epsilon z+A(|z|^2)z+B\overline z^{q-1},$$
if we ignore the terms of degree $q+1$ and higher over $|z|$.
\end{enumerate}

The same conclusions result from the application of averaging methods; however
the normal form is interpreted as the averaged system, describing a slow
evolution of phase curves near the closed one.

Upon use of all of these results we should keep in mind that the exact function
of the sequence is not, generally speaking, a conversion of a phase flow of some
vector field on a plane and does not transition to itself upon action of any
finite group of symmetries.
The execution of these properties for normal form is one of the indications that
series reducing to normal form diverge in the general case.

Nevertheless, many conclusions that can be made from the qualitative picture on
the phase plane of normal form allow us to make conclusions about the behaviour
of the phase curves of the initial system.
For example, the points of equilibrium of the reduced system (shortened normal
form or averaged system) are controlled by closed phase curves of the initial
system, cycles of the reduced system are controlled by invariant tori in the
initial etc.
At the same time finer effects (like the splitting of separators and appearance
of homoclinical and heteroclinical movements) do not appear in the reduced
system.

In the case where the dimension of the initial phase space is greater than
three, while studying loss of stability with the exit of two eigenvalues onto
the circle we can limit ourselves to the study of the three-dimensional
submapping in phase space on which occur the phenomena described above.
Over the other eigendirections the function of the sequence is a contraction
mapping.

The foundation of the possibility to discard these eigendirections as irrelevant
can be achieved with a referral to the aforementioned Shoshitajshvili's general
theory.\\

\centerline{\textbf{6. Study of normal forms}}
\vspace{1em}
From what the above we can see in what way and in what sense the study of loss
of stability near resonances leads to the study of bifurcations of phase
portraits in two-parametric systems of vector fields on a plane, invariant
relative to rotations by angles of $2\pi/q$ where $q$ is the order of the
resonance.
Although the study should produce in the small region of the origin, not
depending on the parameters.

The size of the region in which the system is being studied is the primary
difference between bifurcation theory and local qualitative theory.
Indeed, just like for an individual system as for the availability of parameters
in the neighbourhood of the point of equilibrium, the system can be studied with
methods of local qualitative theory.
We learn that, for example, for which value of the parameter does the
equilibrium lose stability, and get a phase portrait in a certain region of the
point of equilibrium for each value of the parameter whether it is bifurcational
or not.

Additionally, the size of the indicated neighbourhood approaches zero as the
parameter approaches a bifurcational value, even though it is distinct from zero
at the moment of bifurcation.

With this approach the most important phenomenon remains out of our field of
view: the branching of the new equilibrium or cycle at the moment of
bifurcation.
Indeed, at the most critical moment, there is no branching object yet, and at
the next moment it is already out of the operational area of local theory.

In this manner, for studying bifurcations it is essential to study the phase
neighbourhood not approaching zero as the parameter approaches the critical
value.
In other words, we must examine the neighbourhood that is independent of the
parameter in phase space, but the neighbourhood in the straight product of the
phase space and parameter space.

Phase pictures shown on the attached Fig.~1--5 can be observed near resonances
that specifically were not dependent on the parameters of the neighbourhood of
the origin.
A convenient tool for formulation of answers is the language of so called versal
deformations.

A \textit{deformation} of a vector field refers to a family of fields depending
on parameters, that transform into the given field when the parameters have a
zero value.
Two deformations are called \textit{equivalent} if there exists a homeomorphic
(continuous and bijective) replacement of parameters and a continuously
depending on parameters family of homeomorphisms of phase space,
translating parameters and the family of phase portraits of the first
deformation into parameters and corresponding phase portraits of the second.

If we fix a deformation with a certain parameter space then any mapping of
another parameter space into it induces a new deformation.

A deformation of a vector field is called \textit{monologically versal} if any
other deformation of the same field is equivalent the induced one out of the
given one.
In this manner, a versal deformation is in some sense the largest deformation,
containing in itself all others.

A versal deformation with a finite number of parameters exists not for any
field, but if it exists, were to be found and studied, then it would give rather
complete information about all phenomena and reassemblies occurring for any
deformations of the given field.

Versal deformations are determined by an analogous method for other objects,
for example for vector fields with determined conditions of symmetry (meaning
that deformations and phase homeomorphisms alike, in the definition of
equivalence are symmetric).

As explained above, the problem about loss of stability near resonances of order
$q$ requires study of deformations of vector fields on a plane, invariant
relative to rotations by angles of $2\pi/q$.

\textbf{Theorem.}\quad\textit{Vector fields on a plane, invariant relative to
rotations by angles of $2\pi/q,\quad q\neq2,4$, containing a point of
equilibrium at zero with a zero linear part and non-linear terms in general
form, allow a two-parametric versal deformation in the class of symmetric
fields; the corresponding system of differential equations can be written in the
form
$$\dot z=\epsilon z+Az|z|^2+\overline z^{q-1}$$
($\epsilon$ is the complex parameter).}

In the case where $q=4$ theorem is, seemingly, also true, but has not been
proven.
The cases of strong resonances of order $q=1$ and 2 are studied separately and
lead to the formulas indicated above (see section 2).

The proofs of the formulated results are not simple: there are difficulties
involved with the study of edge cycles.
In the simplest case of $q=1$ these difficulties were overcome by Bogdanov in
the years 1970--1971 [2].
In the year 1974 Takens' preprint appeared in which the hypotheses about types
of reassemblies were announced.

The results of study of resonances of order 3 were also published by Gavrilov
[3].
From these works it is unclear, however, how to overcome the difficulties
associated with the study of a reduced system.
Full proofs of all cases where $q\neq4$ and for part of the cases where $q=4$
were given by Horozov (years 1976--1977).
Proofs were based on non-trivial concepts in algebraic geometry and Riemann
surface theory, developed by Ilyashenko [4].

A more detailed statement of the question of versal deformations can be found in
the overview paper cited above of author [2] in [5] and in the book currently
being prepared for printing, ``Additional chapters on the theory of ordinary
differential equations''.\\

\centerline{LITERATURE}
\vspace{1em}
\begin{enumerate}
\item\textit{Shoshitajshvili A.N.}--Works of a seminar for I.G. Petrovskii, v.
1. M., MSU, 1974.
\item\textit{Arnold V.I.}--Successes in Mathematical Studies, 1972, 27,
\textnumero~5, p. 119.
\item\textit{Gavrilov N.K.}--From book: Study on stability and oscillation 
theory. Yaroslavl, YSU, 1977.
\item\textit{Ilyashenko Y.S.}--Functional analysis and its applications, 1977,
11, \textnumero~2, p. 28.
\item\textit{Arnold V.I.}--Functional analysis and its applications, 1977, 11,
\textnumero~2, p. 1.
\end{enumerate}
\end{document}
