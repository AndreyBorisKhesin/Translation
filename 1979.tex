\documentclass[12pt]{amsart}
\usepackage{geometry}                % See geometry.pdf to learn the layout options. There are lots.
\geometry{letterpaper}                   % ... or a4paper or a5paper or ... 
%\geometry{landscape}                % Activate for for rotated page geometry
%\usepackage[parfill]{parskip}    % Activate to begin paragraphs with an empty line rather than an indent
\usepackage{graphicx}
\usepackage{amssymb}
\usepackage{amsmath}
\usepackage{epstopdf}
\usepackage{color}
\usepackage{textcomp}
\usepackage{tikz}
\usetikzlibrary{arrows}
\DeclareGraphicsRule{.tif}{png}{.png}{`convert #1 `dirname #1`/`basename #1 .tif`.png}
\title{Catastrophe Theory}
%\date{}                                           % Activate to display a given date or no date
%\author{V.I.~Arnold, ~~~~ A.P.~Shapiro}
\author{V.I. Arnold}
\begin{document}
\maketitle
{\fontsize{9pt}{1em}\selectfont
Vladimir Igorovitch Arnold, a doctor of physical and mathematical sciences, is a
professor at Moscow State University.
He studies the theory of differential equations, classical and celestial
mechanics, singularity theory, real algebraic geometry, and more.
He wrote \textit{Ordinary Differential Equations} (1971), \textit{Mathematical
Methods in Classical Mechanics} (1974), \textit{Additional Chapters to the
Theory of Ordinary Differential Equations} (1978), and co-wrote \textit{Ergodic
Problems in Classical Mechanics} with A. Avez (1968).
Arnold is a member of the National Mechanics Committee of the USSR, the vice
president of the Moscow Mathematical Society, an honourary member of the London
Mathematical Society, and a recipient of the Lenin Prize.}\\

Catastrophe theory first appeared in Western print about ten years ago.
Magazines such as ``News Week'' reported the revolution in mathematics as
comparable to the invention of Newton's Calculus.
It was claimed that this new science -- catastrophe theory -- was much more
useful to humanity than mathematical analysis: while Newton's theories allowed
for the study of smooth, continuous processes, catastrophe theory provided a
universal method of study for all abrupt transitions, gaps, and sudden
noticeable changes.
Hundreds of scientific and near-scientific publications appeared.
They applied catastrophe theory to wildly different objects, for example, the
beating of a heart, geometrical and physical optics, embryology, linguistics,
experimental psychology, economics, hydrodynamics, geology, and the theory of
elementary particles.
Among published works about catastrophe theory there are studies about the
effect of alcohol on drivers of various modes of transportation, the sturdiness
of ships, prisoner rebellions, the behaviour of stock market players, modeling
brain activity and mental disorders, and the politics of censorship pertaining
to erotic literature.

In the early 1970's, catastrophe theory quickly became a fashionable, widely
advertised theory.
Its universal pretenses reminded of the pseudo-scientific theories of the
previous century.

Pocket editions of the mathematical papers of the founder of catastrophe theory,
the French mathematician R. Tom, were republished and widely circulated.
This had not happened in the mathematical world since the appearance of
cybernetics, which, like catastrophe theory, mostly advertised itself.

After the admirers of catastrophe theory came more sobering, critical papers.
Some of these were printed in publications intended for the layman, bearing
eloquent titles such as ``But he isn't wearing anything at all!''.
Today there are reviews of papers that are critical of catastrophe
theory.\footnote{Couckenheimer, J. \textit{The catastrophe controversy} -- ``The
Mathematical Intelligencer'', 1978, v. 1, \textnumero1, p. 15-21.}
Today there no unanimity about the value of catastrophe theory.
Below is the personal opinion of the author of the paper.\\

\title{H. Whitney's Singularity Theory}

In 1955, American mathematician Hassler Whitney published his paper ``Mappings
of the plane into the plane''\footnote{Whitney H. Mappings of the plane into the
plane -- ``Ann. Math.'', 1955, v. 62, p. 374-470.}.
It laid the foundation of a new mathematical theory -- singularity theory for
smooth mappings.

To truly understand catastrophe theory, one must first be introduced to elements
of Whitney's singularity theory.
Singularity theory is a grandiose generalization of the study of maxima and
minima of functions.
In Whitney's theory, functions are replaced with mappings, in the simplest case
-- mappings of surfaces into the plane.

The mapping of a surface into the plane is a matching of every point on the
surface with a point on the plane.
If a point on the surface is defined by coordinates $(x_1,x_2)$ on the surface
and a point on the plane is defined by coordinates $(y_1,y_2)$, the mapping is
defined by a pair of functions:
$$y_1=f_1(x_1,x_2),\quad y_2=f_2(x_1,x_2).$$
Mappings of smooth surfaces into the plane are all around us.
In fact, most bodies that surround us are bounded by smooth surfaces.
Visible outlines of bodies are just projections of the surfaces surrounding
those bodies onto the eye's retina.
If we were to take a closer look at the bodies that surround us, for example,
the faces of people, we can study the singularities of the visible outlines.

Whitney noticed that in cases of ``general position'', there are only two types
of singularities.
All more complicated singularities are destoryed at the slightest movement of
the bodies or the direction of projection while singularities of these two types
are stable and are preserved despite small deformations in the mapping.

An example of a singularity of the first type (Whitney named it a fold) is the
singularity of a mapping of the projection of a sphere into the plane at point
on its equator (Fig. 1 (Projection of the surface of a sphere into a horizontal
surface.
A singularity arises on the equator of the sphere, which is called a fold.)).
In local coordinates this singularity is defined by these formulas:
$$y_1=x_1^2,\quad y_2=x_2$$
Unsurprisingly, projecting surfaces of smooth bodies on an eye's retina results
in such a singularity in the common points.
What is surprising is that other from this singularity, the fold, we regularly
encounter exactly one other singularity, but almost never notice it.

Whitney named this second singularity ``the assembly''.
It results from projecting surfaces shown in Fig. 2 (The projection of the
surface $y_1=x_1^3+x_1x_2$ into the horizontal plane.
There is a singularity called Whitney's assembly at 0.
Three inverses of point $p$, $u$, $v$, and $w$, are shown.) into the plane.
This surface is defined by the formula $y_1=x_1^3+x_1x_2$ in a space with
coordinates $(x_1,x_2,y_1)$ and is projected into the horizontal plane
$(y_1,y_2)$.

This way, the mapping is defined in local coordinates by the formulas:
$$y_1=x_1^3+x_1x_2,\quad y_2=x_2.$$

On the horizontal plane of projection forms a half-cubic parabola with a return
point (the peak) in the initial coordinates.
This curve separates the horizontal plane into two distinct parts.
Points in the smaller part each have three inverses (three points in the surface
map to this point).
Points in the larger part each have one.
Upon approaching the curve from the smaller part of the projection, two inverses
merge and disappear (this point contains a singularity, a fold); upon
approaching the peak, all three inverses merge.

Whitney proved that the assembly is stable, so any similar mapping has a
similar singularity in a corresponding similar point (such a singularity, so
that a deformed mapping in corresponding coordinates, near the indicated point
is defined by the same formulas as the orginal mapping near the original point).
Whitney also proved that any singularity of a smooth mapping of a surface into
the plane, if disturbed, will promptly ``collapse'' into fold and assemblies.

In this manner, visible contours of smooth bodies in general position contain
return points in places where the projections contain assemblies and do not
contain other singularities: if we look closely, we can find these return points
in the outlines of every face and body.
Let us examine, for example, the surface of a smooth torus (let us say, an
inflated tire).
A torus is normally drawn as depicted in Fig. 3 (A standard depiction of a
torus.).
If the torus were transparent, we would see its outline, as depicted in Fig. 4
(Singularities of the projection of a torus into a plane: 4 assembly points are
visible.): the corresponding mapping of a torus into a plane contains 4 assembly
points.
In this manner, the ends of the lines of the visible outline in Fig. 3 are the
return points.
In these points the line of the visible outline contains a half-cubic singularity.

Transparent tori are hard to come by.
Let us examine another transparent body, a bottle.
Fig. 5 (Bottleneck.
Two assembly points are visible.) depicts a bottleneck; two assembly points are
visible.
By rocking the bottle, we can convince ourselves that the assemblies are stable.
Thus we obtain convincing experimental confirmation of Whitney's theorem.
\end{document}
